\chapter{Parametrización en reglas}
\label{chap:solucion}

\section{Elección del Motor a Utilizar}
Como se mencionó anteriormente, el objectivo de este trabajo es facilitar la gestión de las reglas de un \SIOSU, particularmente por personas sin o con reducidos conocimientos relacionados con la programación. Teniendo esto en cuenta, se considera que, irrespectivamente de los demás aspectos considerados en el criterio de decisión, Drools y OpenL Tablets resultan las únicas opciones de interés para este trabajo. La siguiente es una comparación más detallada de estas opciones:

\paragraph{Expresividad.}
La expresividad que ambos proveen para las reglas resulta comparable. Dicho esto, las tablas de Excel probablemente resulten más familiares para la mayoría de personas que los diagramas DMN.

\paragraph{Gestión de reglas.}
En este aspecto, OpenL Tablets resulta superior, ya que cuenta con herramientas para el versionado de las reglas, además del resto de capacidades que comparte con Drools, como la creación, modificación y eliminación de reglas.

\paragraph{Integración.}
En este campo los motores son similares, por lo menos en su especificación. Dados los problemas mencionados en la subsección de Drools del cápitulo anterior, estás capacidades no pudieron ser probadas para este.

\paragraph{Mantenimiento.}
Ambos proyectos son actualemente (hasta la última fecha de edición de esta sección, 08/04/2025) mantenidos.

Por otra parte, mientras que en el caso de OpenL Tablets no se encontró ningún problema durante la elaboración del ejemplo, lo mismo no se puede decir de Drools. Con este último se encontraron varios problemas, mencionados en la sección correspondiente del cápitulo anterior, relacionados con las herramientas para la creación y edición de las reglas.

\paragraph{Elección final.}
Teniendo en cuenta este análisis, se puede ver que, en los aspectos que resultan de interés para este trabajo, OpenL Tablets resulta igual o superior en cada uno. Consecuentemente, se hará uso del mismo de aquí en adelante.

\section{Caso de estudio 1}

\desarrollar[inline]{Toma el ejemplo más simple desarrollado y explicas como lo parametrizaste con el motor de reglas.}


\section{Caso de estudio 2}

\section{Caso de estudio 3}