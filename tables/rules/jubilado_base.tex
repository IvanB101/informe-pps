\makeatletter\@ifundefined{tablewidth}{\newlength{\tablewidth}}{}
\makeatother\setlength{\tablewidth}{\dimexpr \textwidth - 2\arrayrulewidth - 4\tabcolsep \relax}
\setlength{\extrarowheight}{-5pt}

\begin{tabular}{|p{0.39\tablewidth}|p{0.61\tablewidth}|}
\hline
\multicolumn{2}{|C{{\dimexpr 1.0\tablewidth + 1\arrayrulewidth + 2\tabcolsep \relax}}|}{\color[HTML]{FFFFFF}\cellcolor[HTML]{000000}Rules Numero CalcularJubiladoBase(Afiliacion afiliado, Sistema sistema)}\\ \hline
\color[HTML]{FFFFFF}\cellcolor[HTML]{CCFFFF}C1
	& \color[HTML]{FFFFFF}\cellcolor[HTML]{CCFFFF}RET1\\ \hline
\color[HTML]{000000}\cellcolor[HTML]{CCFFFF}haberPercibido < cuotaMinimaJubilado
	& \cellcolor[HTML]{CCFFFF}\\ \hline
\cellcolor[HTML]{CCFFFF}
	& \color[HTML]{000000}\cellcolor[HTML]{CCFFFF}Numero monto\\ \hline
\color[HTML]{FFFFFF}\cellcolor[HTML]{FFFF99}Sub Categoria
	& \color[HTML]{FFFFFF}\cellcolor[HTML]{FFCC99}Monto\\ \hline
\color[HTML]{000000}\cellcolor[HTML]{FFFF99}yes
	& \color[HTML]{FFFFFF}\cellcolor[HTML]{FFCC99}=cuotaMinimaJubilado\\ \hline
\color[HTML]{000000}\cellcolor[HTML]{FFFF99}no
	& \color[HTML]{FFFFFF}\cellcolor[HTML]{FFCC99}=0.07 * jubilacionMinima + 0.05 * abs(jubilacionMinima - haberPercibido)\\ \hline
\end{tabular}
