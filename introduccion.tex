\chapter{Introducción}

\modificar[inline]{Revisar, modificar y ubicar cada uno según creas conveniente.}

Este proyecto plantea avanzar hacia SIOSU-Moambue, un sistema informático (SI) para obras sociales universitarias (OSU) con características ágiles y de centricidad en el afiliado. La agilidad procura facilitar la evolución en respuesta a cambios y la adaptación/adopción por distintas OSU del país a un costo accesible.  La centricidad en el afiliado (paciente) es la característica preponderante en las innovaciones introducidas recientemente en los sistemas informáticos de salud a partir de avances en distintas tecnologías (Salud 4.0).

El SI de una OSU debe mantenerse “vivo”. Su utilidad, el valor que genera su funcionamiento, disminuye conforme pierde sintonía con los cambios que tanto la organización (por ejemplo, cambios en políticas y reglamentos locales) como su contexto sufren (como ser, cambios en leyes nacionales y provinciales, situaciones sociales o económicas a las que se debe atender, o eventos como Covid-19). Dado que el cambio es la norma, no la excepción, la sistemática falta de evolución de un SIOSU representa su agonía y eventual muerte, e innumerables perjuicios para la OSU y sus afiliados. Inicialmente los funcionarios adaptan sus procesos de trabajo para responder a las modificaciones (posiblemente usando planillas electrónicas o papel), incrementando sistemáticamente la parte manual del trámite en detrimento de la automática. Conforme el estado agónico se profundiza, los funcionarios comienzan a verse abrumados por el trabajo y perciben poco práctico usar las funcionalidades desactualizadas del SIOSU; los trámites que podrían resolverse en minutos, comienzan a llevar horas, días o meses.  Además, los responsables de la gestión experimentan la falta de información actualizada para respaldar sus decisiones, las cuales en ocasiones no pueden demorarse debido a la necesidad de responder a proveedores y prestadores de servicios que podrían interrumpir sus servicios. Por otro lado, el descontento entre los afiliados crece;  perciben que sus trámites no avanzan; que no tienen respuestas por parte de los funcionarios, pero estos a su vez no disponen del tiempo y posiblemente tampoco de las funcionalidades correctas del SIOSU para hacerlo. 

También debe ser tenida en cuenta la revolución innovadora Salud 4.0. Esta se enmarca en la guía de Objetivos de Desarrollo Sostenibles 2030 planteada por la ONU, cuyo tercer objetivo es justamente “Salud y Bienestar”. En general, se busca introducir cambios políticos, económicos, sociales y culturales en diversas esferas de la vida y actividad humana que representen avances catalizadores del progreso y transformación digital de la economía, de los sistemas de producción y de la mejora de la calidad de vida de la sociedad y su respeto a los ecosistemas. En particular, Salud 4.0 nuclea las innovaciones que buscan centrarse en el paciente (impulsadas principalmente por COVID-19) de manera de brindar una atención personalizada que facilite, por ejemplo, detectar anticipadamente y prevenir patologías. Estas innovaciones las posibilitan principalmente los avances en el tratamiento masivo de datos, la ciencia de datos, la inteligencia artificial, los sistemas ciber-físicos, la computación en la nube, y los dispositivos móviles. 

Sin duda es conveniente para las OSU adoptar estas innovaciones. No solo permiten cuidar mejor la salud de sus afiliados y prevenir procedimientos riesgosos y costosos, sino que también se evitan visitas innecesarias a oficinas. De esta manera, se logra la reducción de  costos operativos y se mejoran las posibilidades de inversiones futuras que produzcan  mejoras más profundas en la atención a los afiliados.

Sin embargo, a pesar de los beneficios mencionados, las OSUs difícilmente evitan que sus SIs entren en agonía y en general no abordan innovaciones Salud 4.0 (las cuales son únicamente adoptadas por selectos proveedores de salud privada, muchas veces inaccesible para la población general). Las opciones que tienen las OSU para sacar a sus SIs del camino de la agonía pueden, en general, clasificarse en tres: (a) incorporar/escalar su propia área de informática de manera que esté a la altura del desarrollo y mantenimiento de software necesarios; (b) contratar a una empresa para que esté a cargo de este desarrollo y mantenimiento; y (c) comprarle a una empresa el producto ya desarrollado y pagar por su adaptación y mantenimiento. Es preciso señalar en este punto que las opciones (a) y (b) implican que la OSU sea propietaria del sistema que quieren revitalizar. También se debe estudiar con mucho cuidado la opción (c), el reemplazo del sistema agonizante por uno propietario “enlatado”, que a primera vista puede parecer la opción menos onerosa y más rápida de concretar. La realidad de las OSU es notablemente más compleja que la de las obras sociales privadas. Sus afiliados tienen mayor poder de decisión, muchas veces a través de asambleas. Esto hace que los tipos de afiliados, las reglas de negocio, y los planes de salud sean más equitativos, pero a su vez más complejos. La complejidad también es mayor en la gestión de convenios con prestadores. Al tener un volumen de afiliados reducido, la OSU debe negociar convenios más detallados y asumir la complejidad que las organizaciones prestadoras impongan. Para llevar adelante la opción (c) se deberá pagar la licencia del SI, y luego meses de desarrollo para poder adaptar el mismo a las reglas de negocio de la OSU. Una vez puesto en producción, se deberá conservar la relación con la empresa proveedora, abonando diariamente el mantenimiento y evolución del SI. Además, si se tiene en cuenta que la exportación de servicios informáticos de Argentina representa el 9\% de las exportaciones de servicios, y que los profesionales con la capacidad de llevar adelante el desarrollo necesario, en general cotizan su trabajo a valor dólar, es posible entender que cualquiera de las opciones se tornan inviables para las OSU de menor envergadura, y difíciles de llevar adelante para las de mayor volumen de afiliados, pudiendo incluso no lograr responder a las necesidades de cambio a pesar de las erogaciones y esfuerzo realizados.

La Dirección de Obra Social para el Personal Universitario (DOSPU) de la UNSL no escapa a este problema. La presidencia de DOSPU, el Rectorado, la Facultad de Ciencias Físico-Matemáticas y Naturales, y su Departamento de Informática se encuentran colaborando para reemplazar su antiguo SI que agoniza, y en el proceso causa perjuicios – como los antes enumerados – a la organización. El nuevo sistema, SI-DOSPU, está siendo desarrollado usando prácticas de las metodologías ágiles (scrum, product discovery, behaviour driven development, test driven development, e integración y despliegue contínuo) y sobre  tecnologías de punta incluyendo las que posibilitan desplegar en proveedores de computación en la nube. Este esfuerzo se ha dividido en tres etapas con los siguientes objetivos: (1) reemplazar el antiguo SI y automatizar procesos; (2) dar soporte a la toma de decisiones de la gestión a partir de la creación de un repositorio para el análisis de información y extracción de conocimiento; (3) introducir mejoras en los procesos con el objetivo de centrarlos en los afiliados. Actualmente SI-DOSPU se encuentra en su etapa (1) con la mayor parte de la funcionalidad implementada y en un proceso de carga de datos previos a la puesta en producción. Cabe destacar que el mayor inconveniente que se ha encontrado en el desarrollo de la etapa (1) radica en la resistencia cultural al cambio, a pesar del escaso valor que proporciona el sistema agonizante y gran esfuerzo adicional que esto implica para los funcionarios.

SIOSU-Moambue se desarrollará a partir de SI-DOSPU y la experiencia ganada en su producción. Las OSU difieren en las políticas y reglamentaciones locales, pero es de esperarse que compartan gran parte de su conceptualización y mecanismos de trabajo. SIOSU-Moambue se desarrollará separando las reglas de negocio dependientes de políticas y reglamentaciones locales de manera de hacerlas flexibles y fácilmente modificables. El sistema será liberado con licencia open source y la OSU que quisiera adoptarlo deberá afrontar los gastos de adaptación y mantenimiento locales. Vale la pena destacar que se ha elegido la palabra guaraní “Moambue” que significa cambiar, ya que el foco de la iniciativa es que las OSU puedan evolucionar de manera ágil su SI para responder a los cambios a costos accesibles.

\section{Motivación}

\desarrollar[inline]{En esta sección, explicar el problema que enfrenta la organización al tener reglas complejas: muchas categorías, resoluciones con condiciones a contemplar. Y por otro lado, economía y condiciones sociales inestables.}

Durante su existencia, cada \acrfull{osu}, al igual que tantos otros tipos de organizaciones, se ve afectada por una plétora de cambios, que pueden ser tanto internos (por ejemplo, cambios en políticas y reglamentos), como en su entorno (como ser, cambios en leyes nacionales y provinciales, situaciones sociales o económicas a las que se debe atender, o eventos como Covid-19).
Indiferentemente de la naturaleza de dichos cambios, un \acrfull{si} que fallé en adaptarse a los mismos estará destinado a la obsolencia y eventual muerte, que suele ser precedida por un proceso de degradación del mismo.
Los usuarios del \SIOSU se ven obligados a suplir las funciones desactualizadas con trabajo manual (posiblemente haciendo uso de planillas electrónicas o papel), deshaciendo la automatización y degradando la eficiencia conforme los nuevos requerimientos se separan de aquellos para los que el sistema fue disñado.
Esto, a su vez, conlleva en un aumento de los tiempos requeridos para la realización de los trámites de la \acrshort{osu}, con los subsecuentes decrementos en la calidad del servicio prestado y la conformidad de los afiliados. Adicionalmente, la información se dispersa en los medios introducidos (planillas electrónicas o papel, por ejemplos), dificultando la recolección de información para el respaldo de decisiones.

Esta situación no es más que exacerbada en el caso de que la obra social cuente con reducido o carezca de personal técnico, capaz de trabajar para mantener el sistema consistente con la realidad.

Por otra parte, en el caso de que sea posible realizar los cambios necesarios al sistema, dichos cambios suelen requerir un re-despliegue del sistema, resultando en una menor disponibilidad del mismo. Esto resulta particularmente ineficiente en escenarios donde puede haber frecuentes cambios pequeños.

\section{Objetivo general}
Parametrizar el \SIOSU de \acrfull{dospu}, separando de las reglas de negocio de una obra social en particular del \acrshort{si}, utilizando para la escritura de las mismas un lenguaje inteligible para el personal de la obra social. Más concretamente, el alcance de este trabajo abarca la extracción de las reglas utilizadas para el cálculo de las cuotas de los afiliados.

Esto con el fin de facilitar, que los cambios puedan ser realizados por personal que no tenga conocimientos del funcionamiento interno del \acrshort{si}, facilitando la actualización de las reglas y permitiendo la reducción en la incidencia a o posiblemente una solución a los problemas mencionados.

\section{Objetivos específicos}
\begin{itemize}
	\item Extraer las reglas de negocios del código del \acrlong{si}.
	\item Expresar dichas reglas en un lenguaje que resulte entendible para el personal de la \acrlong{osu}
	\item Permitir la gestión de las reglas de forma independiente del \acrshort{si}.
	\item Reducir la cantidad de esfuerzo requerido para realizar cambios en las reglas.
\end{itemize}

\section{Enfoque adoptado}

\desarrollar[inline]{
    En esta sección explicas los pasos que seguiste, sin entrar en detalles.
    Debe quedar claro: 
    la (auto) capacitación (temas nuevos),
    el trabajo de análisis necesario,
    como se asegura la calidad, y 
    te debe dar la base para explicar en capítulo se desarrolla cada paso.
    }

\section{Organización}

\desarrollar[inline]{Breve descripción de lo tratado por cada capítulo}

Este informe está dividido en los siguientes capítulos:
\begin{itemize}
	\item Gestión de afiliados
	\item Revisión literia
\end{itemize}
