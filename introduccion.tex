\chapter{Introducción}

\modificar[inline]{Revisar estos tres párrafos, modificar y ubicar cada uno según creas conveniente.}

\section{Motivación}

\desarrollar[inline]{En esta sección, explicar el problema que enfrenta la organización al tener reglas complejas: muchas categorías, resoluciones con condiciones a contemplar. Y por otro lado, economía y condiciones sociales inestables.}

Durante su existencia, cada \acrfull{osu}, al igual que tantos otros tipos de organizaciones, se ve afectada por una plétora de cambios, que pueden ser tanto internos (por ejemplo, cambios en políticas y reglamentos), como en su entorno (como ser, cambios en leyes nacionales y provinciales, situaciones sociales o económicas a las que se debe atender, o eventos como Covid-19).
Indiferentemente de la naturaleza de dichos cambios, un \acrfull{si} que fallé en adaptarse a los mismos estará destinado a la obsolencia y eventual muerte, que suele ser precedida por un proceso de degradación del mismo.
Los usuarios del \SIOSU se ven obligados a suplir las funciones desactualizadas con trabajo manual (posiblemente haciendo uso de planillas electrónicas o papel), deshaciendo la automatización y degradando la eficiencia conforme los nuevos requerimientos se separan de aquellos para los que el sistema fue disñado.
Esto, a su vez, conlleva en un aumento de los tiempos requeridos para la realización de los trámites de la \acrshort{osu}, con los subsecuentes decrementos en la calidad del servicio prestado y la conformidad de los afiliados. Adicionalmente, la información se dispersa en los medios introducidos (planillas electrónicas o papel, por ejemplos), dificultando la recolección de información para el respaldo de decisiones.

Esta situación no es más que exacerbada en el caso de que la obra social cuente con reducido o carezca de personal técnico, capaz de trabajar para mantener el sistema consistente con la realidad.

Por otra parte, en el caso de que sea posible realizar los cambios necesarios al sistema, dichos cambios suelen requerir un re-despliegue del sistema, resultando en una menor disponibilidad del mismo. Esto resulta particularmente ineficiente en escenarios donde puede haber frecuentes cambios pequeños.

\section{Objetivo general}
Parametrizar el \SIOSU de \acrfull{dospu}, separando de las reglas de negocio de una obra social en particular del \acrshort{si}, utilizando para la escritura de las mismas un lenguaje inteligible para el personal de la obra social. Más concretamente, el alcance de este trabajo abarca la extracción de las reglas utilizadas para el cálculo de las cuotas de los afiliados.

Esto con el fin de facilitar, que los cambios puedan ser realizados por personal que no tenga conocimientos del funcionamiento interno del \acrshort{si}, facilitando la actualización de las reglas y permitiendo la reducción en la incidencia a o posiblemente una solución a los problemas mencionados.

\section{Objetivos específicos}
\begin{itemize}
	\item Extraer las reglas de negocios del código del \acrlong{si}.
	\item Expresar dichas reglas en un lenguaje que resulte entendible para el personal de la \acrlong{osu}
	\item Permitir la gestión de las reglas de forma independiente del \acrshort{si}.
	\item Reducir la cantidad de esfuerzo requerido para realizar cambios en las reglas.
\end{itemize}

\section{Enfoque adoptado}

\desarrollar[inline]{
    En esta sección explicas los pasos que seguiste, sin entrar en detalles.
    Debe quedar claro: 
    la (auto) capacitación (temas nuevos),
    el trabajo de análisis necesario,
    como se asegura la calidad, y 
    te debe dar la base para explicar en capítulo se desarrolla cada paso.
    }

\section{Organización}

\desarrollar[inline]{Breve descripción de lo tratado por cada capítulo}

Este informe está dividido en los siguientes capítulos:
\begin{itemize}
	\item Gestión de afiliados
	\item Revisión literia
\end{itemize}
