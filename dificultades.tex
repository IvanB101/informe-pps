\section{Dificultades}
Parte de las dificultades se debieron a la falta de costumbre trabajando en projectos de mayor tamaño, de los cuales ha participado un número considerable de personas. Durante la carrera la mayoría de programas y aplicaciones desarrollados los eran de forma individual o en pequeños grupos, requiriendo poco esfuerzo para lograr el entendimiento de gran parte o la totalidad de la base de código de un proyecto. Esta tarea toma un tiempo prohibitavamente largo (por menos inicialmente), en proyectos de mayor tamaño. La solución es simple, utilizar un modelo mental con mejores abstraciones, que no requieran el entendimiento detallado de la totalidad del sistema.

De forma similar, la lectura de los documentos con las reglamentaciones de la obra social, así como los cálculos de las cuotas de la misma están escritos en formatos y con vocabularios un poco distantes de lo que acostumbraba, requiriendo esfuerzos adicionales para la correcta compresión de los mismos.

