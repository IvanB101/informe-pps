\section{Tecnologías empleadas en \SIOSU}\label{sec:tecnologias}

Las tecnologías utilizadas en el proyecto se agrupan de acuerdo a su propósito en: implementación, desarrollo y despliegue y mantenimiento. 

\subsection{Implementación}
\desarrollar[inline, disable]{Tecnologías usadas en la implementación: bd, lógica del negocio, frontend, lenguajes de implementación.}

\subsubsection{Base de Datos}
Una base de datos es una recopilación organizada de información o datos
estructurados. En el contexto del software dichos datos son almacenados de forma electrónica y son administrados por un \acrfull{dbms}. La base de datos en conjunto con el \acrshort{dbms} se denominan sistema de base de datos. En la práctica, el término base de datos suele utilizarse también para referirse a los sistemas de base de datos y a los \acrshort{dbms}.

\paragraph{\href{https://www.postgresql.org/}{PostgreSQL}}
Este el \acrshort{dbms} utilizado en el \SIOSU. Es un sistema de gestión de bases de datos relacionales de código abierto, siendo actualmente uno de los más utilizados y considera por muchos el más avanzado entre los disponibles.

PostgreSQL soporta transacciones que poseen las propiedades \acrfull{acid}, esto en conjunción con otros mecanismos de respaldo, recuperación e integridad de datos permiten evitar la pérdida de información en caso de ocurrir fallos.

Otras capacidades del sistema incluyen:
\begin{description}
    \item[Hot-Standby:] permite acceso de solo lectura cuando se encuentra en modo recuperación o espera.
    \item[Trabajadores Dinámicos] los trabajadores pueden ser iniciados por procesos de usuario.
    \item[Creación de índices no bloqueantes] la creación de índices no resulta en el bloqueo de la tabla correspondiente.
\end{description}
Estas capacidades y muchas otras pueden encontrarse en la documentación \cite{postgreFeatures}.

\subsubsection{Frameworks}
Los frameworks en el contexto del desarrollo de software son abstracciones que proveen funcionalidad genérica, la cual puede ser extendida por los desarrolladores para la creación de aplicaciones.

Para lograr esto suelen proveer un conjunto de bibliotecas, estándares y patrones de diseño, particulares a las tareas para las que el framework esté diseñado.

Adicionalmente, estos imponen restricciones sobre la estructura del código de las aplicaciones, con el fin de lograr una organización más coherente del mismo y promover buenas prácticas de desarrollo.

\paragraph{\href{https://spring.io/projects/spring-boot}{Spring Boot}}\label{para:spring}
Spring Boot es un framework de aplicación de código abierto, utilizando de fondo \href{https://spring.io/}{Spring}. Spring Boot provee infraestructura y valores predefinidos para las configuraciones de Spring, resultando en una configuración mínima de mucho menor tamaño que una aplicación que utiliza Spring directamente. Adicionalemnte, dichos valores pueden luego ser modificados de forma acorde con las necesidades impuestas por la evolución de la aplicación.
Spring Framework está dividido en diversos módulos que podemos utilizar, ofreciéndonos muchas más funcionalidades:
\begin{description}
    \item[Core container:] proporciona inyección de dependencias e inversión de control.
    \item[Web:] nos permite crear controladores Web, tanto de vistas \acrfull{mvc} como aplicaciones \acrfull{rest}.
    \item[Acceso a datos:] abstracciones sobre \acrfull{jdbc}, \acrfull{orm} como Hibernate, \acrfull{oxl}, \acrfull{jms} y transacciones.
    \item[\acrfull{aop}] ofrece el soporte para aspectos.
    \item[Instrumentación:] proporciona soporte para la instrumentación de clases.
    \item[Pruebas de código:] contiene un framework de testing, con soporte para JUnit y TestNG y todo lo necesario para probar los mecanismos de Spring.
\end{description}
Otra característica fundamental de Spring es la ionyección de dependencias, que nos permite definir las dependencias en términos de interfaces, reduciendo así el acoplamiento. Adicionalmente, Spring nos permite cambiar la implementación de una interfaz en función, por ejemplo, de los perfiles actualmente activos.

% TODO: REVISAR: usa Angular o AngularJS.
\paragraph{\href{https://angularjs.org/}{AngularJS}} 
AngularJS (también conocido como Angular 1) es un framework de código abierto escrito en Javascript para el desarrollo de \acrfull{spa}. Desarrollado por Google, ha sido fue discontinuado en enero de 2022 \cite{angularJSDiscontinued} en favor de \href{https://angular.dev/}{Angular}, aunque continúa siendo utilizado en aplicaciones desarrolladas de la descontinuación. 

Los principales objetivos de AngularJS son
\begin{itemize}
    \item Desacoplar la lógica de la aplicación de la manipulación del \acrfull{dom}
    \item Desacoplar el lado del servidor y el lado del cliente de la aplicación.
    \item Proveer una estructura para el desarrollo de las aplicaciones.
\end{itemize}

Angular provee un marco de trabajo para arquitecturas \acrshort{mvc} y \acrfull{mvvm} y la separación de la aplicación en componentes reutilizables, con el fin de simplificar el desarrollo y testeo de \acrshort{spa}.

\paragraph{\href{https://angular.dev/}{Angular}}
Angular (también conocido como Angular 2+ o Angular 2) es un frame escrito en Typescript, de código abierto, utilizado para la creación y mantenimiento de \acrfull{spa}. Desarrollado y mantenido por Google para reemplazar a \href{https://angularjs.org/}{AngularJS} (también conocido como Angular 1), que fue descontinuado en enero de 2022 \cite{angularJSDiscontinued}.

Angular utiliza una arquitectura basada en componentes e inyección de dependencias para mantener el código de la aplicación modular y con acoplamiento reducido. También brinda soluciones para el enrutamiento, formulares y otros tópicos comunes en \acrshort{spa}s, así como soporte para \acrfull{ssr}, \acrfull{ssg}, hidratación del \acrshort{dom} y la división declariva de templates en partes que pueden ser cargadas solamente cuando son necesarias (lazy-loading).

Todo esto permite no solamente un aumento en la productividad de los desarrolladores, sino también en el rendimiento de las aplicaciones, requiriendo menos esfuerzo para que las mismas posean un rendimiento adecuado.

\subsubsection{Lenguajes de Programación}
\paragraph{\href{https://www.java.com/es/}{Java}}
Es un lenguaje de programación orientado a objetos de tipado fuerte y estático desarrollado originalmente por Sun Microsystems y posteriormente adquirido por Oracle. Java continúa siendo de los lengueajes de programción más utilizados el día de hoy \cite{devSurvey2024}.

Uno de sus objetivos de diseño es la reducción en la complejidad para el programador, lo que resulta en una reducción del esfuerzo necesario para el desarrollo \cite{eckelJava}. En adición, esto también reduce el tiempo y esfuerzo necesarios para comenzar a ser productivo con el lenguaje.

Por otra parte, los programas de Java son ejecutados sobre una \acrfull{jvm}, con lo cual cualquier dispositivo que cuente con una puede ejecutar programas escritos en Java.

Por último, dado el tiempo desde su creación y su popularida, existen multitudes de herramientas, frameworks y librerias que asisten en el desarrollo de aplicaciones con este lenguaje. Algunas de estas son Maven (\cref{para:maven}), Spring y Spring Book (\cref{para:spring}), JUnit (\cref{para:junit}) y Mockito (\cref{para:mockito}).

\paragraph{\acrshort{html}}\label{para:html}
\acrfull{html} es el lenguage utilizada para la especificación de la estructura del contenido web mediante el uso de etiquetas.

\acrshort{html} suele ir acompañado de otras tecnologías, notablemente \acrshort{css}(\cref{para:css}) para alterar la presentación del contenido y Javascript (\cref{para:js}) para modificar el comportamiento de las páginas web.

Al momento de redacción de este informe, este lenguaje está regido por el estándar del \acrfull{w3c} \cite{w3cHTML}. Esta naturaleza estandarizada garantiza la consistencia y la interoperabilidad en diferentes navegadores web y plataformas. Esto es importante para crear una experiencia de usuario consistente y accesible \cite{duckettHTMLCSS}.

\paragraph{CSS}\label{para:css}
\acrfull{css} es el lenguaje es el lenguaje de estilos utilizado para describir la presentación de documentos \acrshort{html} o \acrfull{xml} (incluyendo varios lenguajes basados en \acrshort{xml} como \acrfull{svg}, \acrshort{mathml} o \acrshort{xhtml}).

\acrshort{css} utiliza bloques de declaraciones, que definen cambios en los atributos que especifican la presentación de los elementos, en conjunto con selectores, que permiten seleccionar las etiquetas sobre las que serán aplicados los bloques de declaraciones. Estos últimos nos permiten realizar la selección, por ejemplo, mediante el tipo y/o la clase de las etiquetas.

En casos donde hay declaraciones conflictivas para una etiqueta, \acrshort{css} describe un esquema prioritario para decidir que estilo será aplicado. Este esquema es llamado especificidad y está regido por un estándar del \acrshort{w3c} \cite{w3cCSS}.

\paragraph{Javascript}\label{para:js}
\acrfull{js} es un lenguage de programción de alto nivel, generalmente ejecutado utilizando compilación \acrfull{jit}, conforma al estándar ECMAScript \cite{ecmascript} y es actualmente el lenguage de programación más utilizado \cite{devSurvey2024}. Sus características incluyen: tipado dinámico, orientación a objetos basada en prototipos y funciones de primera clase (funciones como valores).

\acrshort{js} es utilizado para modificar el comportamiento de las páginas web, para definir la acción a realizar al presionar un botón, por ejemplo. Junto con \acrshort{css} (\cref{para:css}) y \acrshort{html} (\cref{para:html}), este lenguage forma la base de lo que how es la web.
% TODO: referencia a JSDoc o typescript si se usan en el proyecto

% TODO: el proyecto usa typescript o solo javascript
\paragraph{\href{https://www.typescriptlang.org/}{Typescript}}
\acrfull{ts} es un lenguage de programación de alto nivel agrega tipado estático a \acrshort{js} (\cref{para:js}). Signaturas de funciones, interfaces, genéricos y demás adiciones permiten la especificación de restricciones adicionales, sobre los tipos de los parámetros y retornos de una función, a modo de ejemplo. Esto permite que muchos errores que normalmente aparecerían en tiempo de ejecución sean detectados en tiempo de compilación.

\acrshort{ts} es un superconjunto de \acrlong{js}, con lo cual un programa válido en \acrshort{js} también lo es en \acrshort{ts}, y es transpilado al mismo antes de la ejecución, resultando en la compatibilidad con cualquier entorno que acepte \acrlong{js}.

Por otra parte, la información adicional brindada por las definiciones de tipos también puede ser utilizadas por los servidores de \acrfull{lsp} utilizados por los editores, mejorando no solo los diagnósticos, sino también las sugerencias y el accesso a la documentación dentro del editor.

\subsection{Desarrollo}
\desarrollar[inline, disable]{Tecnologías usadas durante el desarrollo, por ej. gestión del proyecto, control de versiones, aseguramiento de la calidad, evolución de la base de datos ...}

\paragraph{\href{https://maven.apache.org/}{Maven}}\label{para:maven}
Es una herramienta de software para la gestión de la compilación, testeo y documentación de proyectos. Maven hace uso de un \acrfull{pom} para describir el proyecto, aspectos tales como sus dependencias de otros módulos y componentes externos, aspectos de compilación y testeo tanto del proyecto como de las dependencias y los plugins utilizados.

Otra característica de suma utilidad es el manejo de dependencias. Maven permite la descarga de repositorios, como el repositorio central de Maven, de forma automática al compilar proyectos con dependencias especificadas en el POM. Adicionalmente, son resueltas las dependencias transitivas (dependencias de dependencias), evitando la necesidad de especificar la totalidad de las dependencias \cite{sonatypeMaven}.

Por otra parte, esta herramienta nos permite el empaquetado en formatos como \acrfull{jar}, que puede ser utilizado como dependencia por otros proyectos, y \acrfull{war}. En este último caso, se puede configurar las dependencias que serán empaquetadas con el artefacto final.

\paragraph{\href{https://git-scm.com/}{Git}}
Es un sistema de control de versiones distribuido, rápido y escalable, el cual brinda comandos tanto para operaciones de alto nivel como para el acceso a las partes internas del mismo \cite{gitDocs}.

Entre las capacidades de esta herramienta, se encuentra la de mantener varias ramas separadas para el desarrollo concurrente de una aplicación, pudiendo cambiar entre las mismas con gran facilida y luego combinarlas de forma ergonómica.

Además, se pueden utilizar plataformas como \href{https://github.com/}{Github} o \href{https://gitlab.com/}{Gitlab} para alojar repositorios, facilitando la colaboración en los mismos.

\paragraph{\href{https://junit.org/}{JUnit}}\label{para:junit}
JUnit es una framework de automatización de test para Java, para lo cual expone una api con anotaciones. Algunas de las anotaciones más comunes son las siguientes, que se utilizan para anotar funciones:
\begin{description}
    \item[\code{@BeforeAll}] La función se ejecuta una única vez antes de todos los tests de la clase.
    \item[\code{@BeforeEach}] La función se ejecuta antes de cada test de la clase.
    \item[\code{@Test}] Utilizada para las funciones que son los tests.
    \item[\code{@AfterAll}, \code{@AfterEach}] Similar a \code{@BeforeAll} y \code{@BeforeEach} pero ejecutando las funciones después de todos y cada uno de los tests respectivamente.
\end{description}

Este framework también cuenta con integración con herramientas como Maven. Esto nos permite, por ejemplo ejecutar los tests correspondientes a una clase con un comando de Maven, o ejecutar los tests en alguna etapa del ciclo de vida automáticamente.

\paragraph{\href{https://site.mockito.org/}{Mockito}}\label{para:mockito}
Es un marco de trabajo de código abierto para testing, normalmente utilizado en conjunción con \acrfull{tdd} o \acrfull{bdd}.

Uno de los objetivos de este framework es proveer una \acrfull{api} simple e intuitiva para la creación de Objetos Simulados (Mock Objects), que son objetos que imita parte del comportamiento de un objeto del sistema, así como la creación de stubs (o talones) \cite{mockitoFeatures}.

Los objetos simulados son objetos que imita parte del comportamiento de un objeto del sistema Los stubs son métodos que reemplazan alguna funcionalidad. En el contexto del testing, generalmente los stubs definen el comportamiento de los objetos simulados.
A modo de ejemplo, si se está probando una funcionalidad que requiere utilizar una base de datos, se puede utilizar un objeto simulado, que simula la conexión con la misma y las respuestas que esta tiene ante determinadas consultas, con el fin de evitar que la base de datos esté siendo ejecutada para poder ejecutar el test.

De igual forma, Mockito posee otras características como errores de verificación limpios, comparadores de argumentos y la capacidad de crear los propios, entre otras. 

\paragraph{\href{https://cucumber.io/docs}{Cucumber}}
Es una herramienta para facilitar el \acrshort{bdd}. Para lograr esto permite la especificación de escenarios en lenguage natural, teniendo este que seguir un conjunto de reglas de gramática simples llamado Gherkin. Luego, mediante definiciones de paso (step definitions) se puede mapear las acciones utilizadas en el escenario con código a ejecutar. Cucumber permite trabajar con una gran variedad de lenguages de programación, dentro de los cuales se encuentra Java.

Una de las grandes ventajas de esta herramienta, y otras similares, es que, al estar los escenarios especficados en lenguage natural, los mismos pueden ser válidados por las partes interesadas.

\paragraph{\href{https://www.red-gate.com/products/flyway/community/}{Flyway}}
Es una herramienta de código abierto para facilitar las migraciones de bases de datos, al igual que el control de versiones de sus esquemas. Puede ser utilizada para la automatización de migraciones y también integrada en pipelines de \acrshort{ci} y \acrshort{cd}.

Flyway puede ser utilizado PosrgreSQL y MySQL, entre otras, y posee integración con otras herramientas utilizadas para la administración de proyectos, dentro de las cuales se encuentra Maven. 

\subsection{Despliegue y mantenimiento}
\desarrollar[inline, disable]{Tecnologías usadas para el despliegue y manteniemiento, por ej. para la virtualización y monitoreo (docker y kubernetes).}

\paragraph{\href{https://www.docker.com/}{Docker}}
Docker es una tecnología de código abierto para la contenerización de aplicaciones, que consiste en el empaquetado del código de la misma en conjunto con las dependencias necesarias para su ejecución, para crear un único entorno aislado y ligero, llamado contenedor \cite{ibmContainerization}.

Al empaquetar el código junto con las dependencias, tales como librerías y archivos de configuración, se desacopla de la plataforma sobre la que se ejecuta, siendo solamente necesario que esta cuente con Docker.

Distintos contenedores comparten el núcleo del \acrfull{so} host, y además pueden compartir binarios y librerías, razón por la cual son llamados ligeros. Consecuentemente, los contenedores son más ligeros y rápidos para inicial que las máquinas virtuales.

\paragraph{\href{https://www.jenkins.io/}{Jetkins}}
Es un servidor de código para la automatización de tareas de compilación, testing, entrega y despliegue, utilizado para facilitar la \acrfull{ci} y \acrfull{cd}.

\paragraph{\href{https://kubernetes.io/es/}{Kubernetes}}
Es una plataforma de código abierto para la automatización del despliegue, escalado y administración de aplicaciones basadas en contenedores.

Kubernetes agrupa una o más máquinas, que pueden ser virtuales o físicas, llamadas nodos, en un cluster, sobre el cual se pueden ejecutar cargas de trabajo en contenedores. Los contenedores son organizados en pods, que representan la unidad de despliegue y pueden correr sobre los nodos.

Ejemplos de las tareas que pueden ser automatizadas son la replicación de pods cuando la carga pod lo requiera, y la terminación de pods cuando han dejado de funcionar correctamente.
