\documentclass[11pt,a4paper,twoside]{memoir} 

\usepackage[spanish]{babel}

\usepackage[bookmarks=true]{hyperref}

% Agrega toc al toc
\usepackage{tocbibind}

\usepackage[acronym]{glossaries}
\makeglossaries

\newacronym{si}{SI}{Sistema Informático}
\newacronym{osu}{OSU}{Obra Social Universitaria}
\newacronym{dospu}{DOSPU}{Dirección de Obra Social para el Personal Universitario}
\newacronym{decom}{DECOM}{Departamento de Complementación}
\newacronym{fesac}{FESAC}{Fondo Especial Solidario para Alta Complejidad}
\newacronym{sumas}{SUMAS}{Sistema Universitario Médico Asistencial Solidario}
\newacronym{unsl}{UNSL}{Universidad Nacional de San Luis}
\newacronym{cmmu}{CMMU}{Cuota Mensual Máxima Única}
\newacronym{js}{JS}{Javascript}
\newacronym{ts}{TS}{Typescript}
\newacronym{so}{SO}{Sistema Operativo}
\newacronym[
	user1={Decision Requirements Diagram}
]{drd}{DRD}{Diagrama de Requemientos de Decisión}
\newacronym[
	user1={Drools Rule Language}
]{drl}{DRL}{Lenguaje de Reglas de Drools}
\newacronym[
	user1={Friendly Enough Expression Language}
]{feel}{FEEL}{Lenguaje de Expresión Suficientemente Amigable}
\newacronym[
	user1={Decision Model Notation}
]{dmn}{DMN}{Notación de Modelo de Decisión}
\newacronym[ user1={Plain Old Java Object} ]{pojo}{POJO}{Objeto Java Simple}
\newacronym[
	user1={Database Management System}
]{dbms}{DBMS}{Sistema de Gestión de Base de Datos}
\newacronym[
	user1={Atomicity, Consistency, Isolation and Durability}
]{acid}{ACID}{Atomicidad, Consistencia (o Integridad), Aislamiento y Duranbilidad (o Persistencia)}
\newacronym[
	user1={Aspect Oriented Programming}
]{aop}{AOP}{Programación Orientada a Aspectos}
\newacronym[ user1={Model View Controller} ]{mvc}{MVC}{Modelo Vista Controlador}
\newacronym[ user1={Model View Viewmodel} ]{mvvm}{MVVM}{Modelo Vista Modelo de Vista}
\newacronym[
	user1={Representational Data Transfer}
]{rest}{REST}{Transferencia de Estado Representacional}
\newacronym[
	user1={Java Database Connectivity}
]{jdbc}{JDBC}{Conectividad a Base de Datos de Java}
\newacronym[ user1={Object Relational Mapping} ]{orm}{ORM}{Mapeo Relacional de Objectos}
\newacronym[ user1={Object XML Mapping} ]{oxl}{OXL}{Mapeo de Objectos a XML}
\newacronym[ user1={Java Message Service} ]{jms}{JMS}{Servicio de Mensajes de Java}
\newacronym[ user1={Project Object Model} ]{pom}{POM}{Modelo de Objetos de Proyecto}
\newacronym[ user1={Java Archive} ]{jar}{JAR}{Archivo Java}
\newacronym[ user1={Web Application Resource} ]{war}{WAR}{Recurso de Aplicación Web}
\newacronym[ user1={Java Virtual Machine} ]{jvm}{JVM}{Máquina Virtual de Java}
\newacronym[
	user1={HyperText Markup Language}
]{html}{HTML}{Lenguaje de Marcardo de Hipertexto}
\newacronym[
	user1={Estensible HyperText Markup Language}
]{xhtml}{XHTML}{Lenguaje de Marcardo de Hipertexto Extensible}
\newacronym[ user1={Extensible Markup Language} ]{xml}{XML}{Lenguage de Marcado Extensible}
\newacronym[ user1={Scalable Vector Graphics} ]{svg}{SVG}{Gráficos Vectoriales Escalables}
\newacronym[
    user1={Mathematical Markup Language}
]{mathml}{MathML}{Lenguage de Marcado Matemático}
\newacronym[ user1={Cascading Style Sheet} ]{css}{CSS}{Hojas de Estilo en Cascada}
\newacronym[ user1={Single Page Application} ]{spa}{SPA}{Aplicación de página única}
\newacronym[ user1={Document Object Model} ]{dom}{DOM}{Modelo de Objetos del Documento}
\newacronym[ user1={Static Site Generation} ]{ssg}{SSG}{Generación Estática de Sitios}
\newacronym[ user1={Server Side Rendering} ]{ssr}{SSR}{Renderizado del Lado del Servidor}
\newacronym[ user1={World Wide Web Consortium} ]{w3c}{W3C}{Consorcio WWW}
\newacronym[ user1={Just In Time} ]{jit}{JIT}{Justo a tiempo}
\newacronym[ user1={Language Server Protocol} ]{lsp}{LSP}{Protocolo de Servidor de Lenguage}
\newacronym[ user1={Test Driven Development} ]{tdd}{TDD}{Desarrollo Guiado por Pruebas}
\newacronym[
    user1={Behavior Driven Development}
]{bdd}{BDD}{Desarrollo Guiado por Comportamiento}
\newacronym[
    user1={Application Programming Interface}
]{api}{API}{Interfaz de Programación de Aplicaciones}
\newacronym[ user1={Continous Delivery} ]{cd}{CD}{Entrega Continua}
\newacronym[ user1={Continous Integration} ]{ci}{CI}{Integración Continua}


\usepackage[inline]{enumitem}

\usepackage{caption}
\captionsetup[listing]{name=Listado}

\usepackage{cleveref}
\crefname{listing}{listado}{listados}
\Crefname{listing}{Listado}{Listados}

\usepackage{fontspec}
\setmonofont{MesloLGS NF}

\usepackage[]{minted}
\setminted[]{
    breaklines,
    breakafter=(.,
    fontsize=\scriptsize,
    linenos,
    escapeinside=||,
    frame=single,
    rulecolor=\color{black},
    tabsize=2,
}
\setmintedinline[]{
    fontsize=\scriptsize,
}

\usepackage{graphicx}
\graphicspath{ {./images/} }

\usepackage{csquotes}
\usepackage[backend=biber, style=numeric, sorting=none, bibstyle=alphabetic, maxnames=10, minnames=6,backref=true, autocite=inline, labelalpha=true, doi=false, url=false]{biblatex}
\addbibresource{the.bib}
% tablas
\usepackage[margin=1in]{geometry}
\usepackage[table,dvipsnames]{xcolor} 
\usepackage{multirow}
\usepackage{etoolbox}
\usepackage{tabularx}
\usepackage{array}
\newcolumntype{C}[1]{>{\centering\arraybackslash}p{#1}}

% Use more than one optional parameter in a new commands
\usepackage{xargs}                      

\usepackage[spanish,colorinlistoftodos,prependcaption,textsize=tiny]{todonotes}
\newcommandx{\desarrollar}[2][1=]{\todo[linecolor=blue,backgroundcolor=blue!25,bordercolor=blue,#1]{#2}}
\newcommandx{\modificar}[2][1=]{\todo[linecolor=Plum,backgroundcolor=Plum!25,bordercolor=Plum,#1]{#2}}
% \newcommandx{\unsure}[2][1=]{\todo[linecolor=red,backgroundcolor=red!25,bordercolor=red,#1]{#2}}
% \newcommandx{\info}[2][1=]{\todo[linecolor=OliveGreen,backgroundcolor=OliveGreen!25,bordercolor=OliveGreen,#1]{#2}}
% \newcommandx{\improvement}[2][1=]{\todo[linecolor=Plum,backgroundcolor=Plum!25,bordercolor=Plum,#1]{#2}}
% \newcommandx{\thiswillnotshow}[2][1=]{\todo[disable,#1]{#2}}


\newcommand{\SIOSU}{\acrshort{si}--\acrshort{osu}}

\newcommand*{\thistitle}{\begingroup% based on titleJT
	\centering
	\includegraphics[width=0.5\textwidth]{logo-UNSL}\par\vspace{1cm}
	{\scshape\LARGE Universidad Nacional de San Luis \par}
	\vspace{1cm}
	{\scshape\Large Facultad de Ciencias Físico Matemática y Naturales\par}
	\vspace{1.5cm}
	{\huge\bfseries Parametrizando un Sistema Informático para Obra Social Universitaria con un Motor de Reglas \par}
	\vspace{2cm}
	{\Large\itshape Iván Brocas\par}
	\vfill
	{\Large\itshape Director: Mg. Alejandro Sánchez\par}
	{\Large\itshape Co-Director: Dr. Carlos Humberto Salgado\par}
	\vfill
	% Bottom of the page
	{\large \today\par}
	\endgroup}

\newcommand{\code}[1]{{\small\texttt{#1}}}

% para acrónimos provenientes del inglés
\newcommand{\acrcomplete}[1]{
	\acrlong{#1} (\glsentrydesc{#1}, \acrshort{#1})
}


\title{Informe PPS}
\author{Iván Brocas}

\begin{document}

% \thistitle

% Esta lista se borra al momento de distribuir
% \cleardoublepage
% \listoftodos
% \todototoc

% \cleardoublepage
% \chapter*{Resumen}

% \cleardoublepage
%  \renewcommand{\contentsname}{Tabla de contenidos}
% \tableofcontents

\printglossary[type=\acronymtype]

% 8. Selección de caso puntual para prototipar la separación de las reglas de cálculo del
% código fuente
% 9. Implementación de tests de verificación de correcto comportamiento de la aplicación
% para el caso puntual.
% 10. Implementación del cálculo para un caso puntual y verificación del correcto
% comportamiento.

\chapter{Introducción}

\modificar[inline]{Revisar, modificar y ubicar cada uno según creas conveniente.}

Este proyecto plantea avanzar hacia SIOSU-Moambue, un sistema informático (SI) para obras sociales universitarias (OSU) con características ágiles y de centricidad en el afiliado. La agilidad procura facilitar la evolución en respuesta a cambios y la adaptación/adopción por distintas OSU del país a un costo accesible.  La centricidad en el afiliado (paciente) es la característica preponderante en las innovaciones introducidas recientemente en los sistemas informáticos de salud a partir de avances en distintas tecnologías (Salud 4.0).

El SI de una OSU debe mantenerse “vivo”. Su utilidad, el valor que genera su funcionamiento, disminuye conforme pierde sintonía con los cambios que tanto la organización (por ejemplo, cambios en políticas y reglamentos locales) como su contexto sufren (como ser, cambios en leyes nacionales y provinciales, situaciones sociales o económicas a las que se debe atender, o eventos como Covid-19). Dado que el cambio es la norma, no la excepción, la sistemática falta de evolución de un SIOSU representa su agonía y eventual muerte, e innumerables perjuicios para la OSU y sus afiliados. Inicialmente los funcionarios adaptan sus procesos de trabajo para responder a las modificaciones (posiblemente usando planillas electrónicas o papel), incrementando sistemáticamente la parte manual del trámite en detrimento de la automática. Conforme el estado agónico se profundiza, los funcionarios comienzan a verse abrumados por el trabajo y perciben poco práctico usar las funcionalidades desactualizadas del SIOSU; los trámites que podrían resolverse en minutos, comienzan a llevar horas, días o meses.  Además, los responsables de la gestión experimentan la falta de información actualizada para respaldar sus decisiones, las cuales en ocasiones no pueden demorarse debido a la necesidad de responder a proveedores y prestadores de servicios que podrían interrumpir sus servicios. Por otro lado, el descontento entre los afiliados crece;  perciben que sus trámites no avanzan; que no tienen respuestas por parte de los funcionarios, pero estos a su vez no disponen del tiempo y posiblemente tampoco de las funcionalidades correctas del SIOSU para hacerlo. 

También debe ser tenida en cuenta la revolución innovadora Salud 4.0. Esta se enmarca en la guía de Objetivos de Desarrollo Sostenibles 2030 planteada por la ONU, cuyo tercer objetivo es justamente “Salud y Bienestar”. En general, se busca introducir cambios políticos, económicos, sociales y culturales en diversas esferas de la vida y actividad humana que representen avances catalizadores del progreso y transformación digital de la economía, de los sistemas de producción y de la mejora de la calidad de vida de la sociedad y su respeto a los ecosistemas. En particular, Salud 4.0 nuclea las innovaciones que buscan centrarse en el paciente (impulsadas principalmente por COVID-19) de manera de brindar una atención personalizada que facilite, por ejemplo, detectar anticipadamente y prevenir patologías. Estas innovaciones las posibilitan principalmente los avances en el tratamiento masivo de datos, la ciencia de datos, la inteligencia artificial, los sistemas ciber-físicos, la computación en la nube, y los dispositivos móviles. 

Sin duda es conveniente para las OSU adoptar estas innovaciones. No solo permiten cuidar mejor la salud de sus afiliados y prevenir procedimientos riesgosos y costosos, sino que también se evitan visitas innecesarias a oficinas. De esta manera, se logra la reducción de  costos operativos y se mejoran las posibilidades de inversiones futuras que produzcan  mejoras más profundas en la atención a los afiliados.

Sin embargo, a pesar de los beneficios mencionados, las OSUs difícilmente evitan que sus SIs entren en agonía y en general no abordan innovaciones Salud 4.0 (las cuales son únicamente adoptadas por selectos proveedores de salud privada, muchas veces inaccesible para la población general). Las opciones que tienen las OSU para sacar a sus SIs del camino de la agonía pueden, en general, clasificarse en tres: (a) incorporar/escalar su propia área de informática de manera que esté a la altura del desarrollo y mantenimiento de software necesarios; (b) contratar a una empresa para que esté a cargo de este desarrollo y mantenimiento; y (c) comprarle a una empresa el producto ya desarrollado y pagar por su adaptación y mantenimiento. Es preciso señalar en este punto que las opciones (a) y (b) implican que la OSU sea propietaria del sistema que quieren revitalizar. También se debe estudiar con mucho cuidado la opción (c), el reemplazo del sistema agonizante por uno propietario “enlatado”, que a primera vista puede parecer la opción menos onerosa y más rápida de concretar. La realidad de las OSU es notablemente más compleja que la de las obras sociales privadas. Sus afiliados tienen mayor poder de decisión, muchas veces a través de asambleas. Esto hace que los tipos de afiliados, las reglas de negocio, y los planes de salud sean más equitativos, pero a su vez más complejos. La complejidad también es mayor en la gestión de convenios con prestadores. Al tener un volumen de afiliados reducido, la OSU debe negociar convenios más detallados y asumir la complejidad que las organizaciones prestadoras impongan. Para llevar adelante la opción (c) se deberá pagar la licencia del SI, y luego meses de desarrollo para poder adaptar el mismo a las reglas de negocio de la OSU. Una vez puesto en producción, se deberá conservar la relación con la empresa proveedora, abonando diariamente el mantenimiento y evolución del SI. Además, si se tiene en cuenta que la exportación de servicios informáticos de Argentina representa el 9\% de las exportaciones de servicios, y que los profesionales con la capacidad de llevar adelante el desarrollo necesario, en general cotizan su trabajo a valor dólar, es posible entender que cualquiera de las opciones se tornan inviables para las OSU de menor envergadura, y difíciles de llevar adelante para las de mayor volumen de afiliados, pudiendo incluso no lograr responder a las necesidades de cambio a pesar de las erogaciones y esfuerzo realizados.

La Dirección de Obra Social para el Personal Universitario (DOSPU) de la UNSL no escapa a este problema. La presidencia de DOSPU, el Rectorado, la Facultad de Ciencias Físico-Matemáticas y Naturales, y su Departamento de Informática se encuentran colaborando para reemplazar su antiguo SI que agoniza, y en el proceso causa perjuicios – como los antes enumerados – a la organización. El nuevo sistema, SI-DOSPU, está siendo desarrollado usando prácticas de las metodologías ágiles (scrum, product discovery, behaviour driven development, test driven development, e integración y despliegue contínuo) y sobre  tecnologías de punta incluyendo las que posibilitan desplegar en proveedores de computación en la nube. Este esfuerzo se ha dividido en tres etapas con los siguientes objetivos: (1) reemplazar el antiguo SI y automatizar procesos; (2) dar soporte a la toma de decisiones de la gestión a partir de la creación de un repositorio para el análisis de información y extracción de conocimiento; (3) introducir mejoras en los procesos con el objetivo de centrarlos en los afiliados. Actualmente SI-DOSPU se encuentra en su etapa (1) con la mayor parte de la funcionalidad implementada y en un proceso de carga de datos previos a la puesta en producción. Cabe destacar que el mayor inconveniente que se ha encontrado en el desarrollo de la etapa (1) radica en la resistencia cultural al cambio, a pesar del escaso valor que proporciona el sistema agonizante y gran esfuerzo adicional que esto implica para los funcionarios.

SIOSU-Moambue se desarrollará a partir de SI-DOSPU y la experiencia ganada en su producción. Las OSU difieren en las políticas y reglamentaciones locales, pero es de esperarse que compartan gran parte de su conceptualización y mecanismos de trabajo. SIOSU-Moambue se desarrollará separando las reglas de negocio dependientes de políticas y reglamentaciones locales de manera de hacerlas flexibles y fácilmente modificables. El sistema será liberado con licencia open source y la OSU que quisiera adoptarlo deberá afrontar los gastos de adaptación y mantenimiento locales. Vale la pena destacar que se ha elegido la palabra guaraní “Moambue” que significa cambiar, ya que el foco de la iniciativa es que las OSU puedan evolucionar de manera ágil su SI para responder a los cambios a costos accesibles.

\section{Motivación}

\desarrollar[inline]{En esta sección, explicar el problema que enfrenta la organización al tener reglas complejas: muchas categorías, resoluciones con condiciones a contemplar. Y por otro lado, economía y condiciones sociales inestables.}

Durante su existencia, cada \acrfull{osu}, al igual que tantos otros tipos de organizaciones, se ve afectada por una plétora de cambios, que pueden ser tanto internos (por ejemplo, cambios en políticas y reglamentos), como en su entorno (como ser, cambios en leyes nacionales y provinciales, situaciones sociales o económicas a las que se debe atender, o eventos como Covid-19).
Indiferentemente de la naturaleza de dichos cambios, un \acrfull{si} que fallé en adaptarse a los mismos estará destinado a la obsolencia y eventual muerte, que suele ser precedida por un proceso de degradación del mismo.
Los usuarios del \SIOSU se ven obligados a suplir las funciones desactualizadas con trabajo manual (posiblemente haciendo uso de planillas electrónicas o papel), deshaciendo la automatización y degradando la eficiencia conforme los nuevos requerimientos se separan de aquellos para los que el sistema fue disñado.
Esto, a su vez, conlleva en un aumento de los tiempos requeridos para la realización de los trámites de la \acrshort{osu}, con los subsecuentes decrementos en la calidad del servicio prestado y la conformidad de los afiliados. Adicionalmente, la información se dispersa en los medios introducidos (planillas electrónicas o papel, por ejemplos), dificultando la recolección de información para el respaldo de decisiones.

Esta situación no es más que exacerbada en el caso de que la obra social cuente con reducido o carezca de personal técnico, capaz de trabajar para mantener el sistema consistente con la realidad.

Por otra parte, en el caso de que sea posible realizar los cambios necesarios al sistema, dichos cambios suelen requerir un re-despliegue del sistema, resultando en una menor disponibilidad del mismo. Esto resulta particularmente ineficiente en escenarios donde puede haber frecuentes cambios pequeños.

\section{Objetivo general}
Parametrizar el \SIOSU de \acrfull{dospu}, separando de las reglas de negocio de una obra social en particular del \acrshort{si}, utilizando para la escritura de las mismas un lenguaje inteligible para el personal de la obra social. Más concretamente, el alcance de este trabajo abarca la extracción de las reglas utilizadas para el cálculo de las cuotas de los afiliados.

Esto con el fin de facilitar, que los cambios puedan ser realizados por personal que no tenga conocimientos del funcionamiento interno del \acrshort{si}, facilitando la actualización de las reglas y permitiendo la reducción en la incidencia a o posiblemente una solución a los problemas mencionados.

\section{Objetivos específicos}
\begin{itemize}
	\item Extraer las reglas de negocios del código del \acrlong{si}.
	\item Expresar dichas reglas en un lenguaje que resulte entendible para el personal de la \acrlong{osu}
	\item Permitir la gestión de las reglas de forma independiente del \acrshort{si}.
	\item Reducir la cantidad de esfuerzo requerido para realizar cambios en las reglas.
\end{itemize}

\section{Enfoque adoptado}

\desarrollar[inline]{
    En esta sección explicas los pasos que seguiste, sin entrar en detalles.
    Debe quedar claro: 
    la (auto) capacitación (temas nuevos),
    el trabajo de análisis necesario,
    como se asegura la calidad, y 
    te debe dar la base para explicar en capítulo se desarrolla cada paso.
    }

\section{Organización}

\desarrollar[inline]{Breve descripción de lo tratado por cada capítulo}

Este informe está dividido en los siguientes capítulos:
\begin{itemize}
	\item Gestión de afiliados
	\item Revisión literia
\end{itemize}


\chapter{\acrfull{dospu}}

Según \cite{CSOrd17}, la misión de \acrshort{dospu} consiste en:

\begin{displayquote}
Entender en la organización administrativa de la Obra Social y en todos los aspectos contables, financieros y de acción social a cargo de la DOSPU.

Entender en todas las actividades inherentes a la asistencia, altas y bajas del personal de la DOSPU.

Entender en los procedimientos de incorporación, registro y documentación de las y los afiliados.

Entender en todos los aspectos de financiación, manejo de valores y recursos económicos a cargo de la DOSPU.

Entender en la supervisión del sistema de asistencia médica, odontológica, química, bioquímica, farmacéutica, psicológica y demás profesionales de la salud.

Controlar la atención médica según operatoria de Auditoría definida previamente por DOSPU.

Coordinar y organizar la provisión y expendio de los productos farmacéuticos para pacientes crónicos y de alto costo.

Instrumentar y operar las políticas, normas, sistemas y procedimientos necesarios para garantizar la exactitud y seguridad en la captación y registro de las operaciones financieras, presupuestarias y de consecución de metas de DOSPU y DECOM. Llevar el registro contable, de patrimonio, realizar el control presupuestario, refrendar y supervisar la confección de los balances, estados patrimoniales.

Supervisar la asistencia médica, sanitaria integral, sistema de reciprocidad y celebraciones de convenios con prestadores.

Asegurar la prestación de todos los servicios que ofrece la Obra Social al Personal Universitario en el Complejo Universitario de Villa Mercedes.
\hfill\parencite{CSOrd17}
\end{displayquote}

Por otra parte, dado el tamaño de la dirección, esta se encuentra divivida en varias direcciones que, a su vez, son dividas en departamentos (ver \cref{fig:estructura_dospu}). En las siguientes secciones se encuentran las tareas de las que se ocupa cada una de las direcciones y los departamentos de \acrshort{dospu}.

\begin{figure}
    \centering
    \includegraphics[width=\textwidth]{estructura_dospu.png}
    \caption{Estructura organizacional \acrshort{dospu} \cite{CSOrd17}}
    \label{fig:estructura_dospu}
\end{figure}

\section{Dirección General}
\begin{displayquote}
a) Asistir a la Presidencia en la confección de resoluciones, providencias, ordenanzas, notas, comunicaciones y sus respectivos registros.

b) Organizar la administración y servicios con acuerdo del Directorio.

c) Cumplir y hacer cumplir el Estatuto, las reglamentaciones y resoluciones de la Obra Social.

d) Someter a consideración de Presidencia y el Directorio, las contrataciones correspondientes a la prestación de servicios.

e) Colaborar en el diseño de todas las medidas que sean necesarias para la solución de problemas de afiliados que sean de competencia de la Obra Social.

f) Controlar y verificar todo lo concerniente a la contabilidad y control económico-financiero y de la prestación de los servicios.

g) Elevar mensualmente a consideración del Directorio un informe demostrativo de la situación económica-financiera de DOSPU y DECOM, juntamente con referencias estadísticas que permitan orientar la acción a seguir en base a las posibilidades económicas.

h) Ofrecer al Directorio informes demostrativos de la situación económico-financiera de DOSPU y DECOM, con referencias estadísticas que permitan orientar la acción a seguir en base a las posibilidades económicas.

i) Firmar los estados mensuales de caja, cuadros estadísticos, inventarios y balances.

j) Suscribir juntamente con Presidencia la documentación relativa a pagos y promesas de pago, como así también las contrataciones con terceros.

k) Suscribir con Presidencia la Memoria de DOSPU y DECOM.

l) Suscribir los estados mensuales de caja, cuadros estadísticos, inventarios y balances.

m) Solicitar al Directorio la aplicación de sanciones disciplinarias a los afiliados que no cumplan con las disposiciones de DOSPU y DECOM.

n) Entender en la representación de DOSPU ante otras Obras Sociales con las que se firmen convenios de Reciprocidad, a pedido de Presidencia.

o) Conservar los antecedentes y documentación de los acuerdos del Directorio.

p) Intervenir en la difusión, resolución y ejecución de todos los asuntos relativos a los intereses de la obra social.

q) Desempeñar toda otra actividad que la autoridad le encomiende, en el área de trabajo y de
acuerdo con la responsabilidad específica.
\hfill\parencite{CSOrd17}
\end{displayquote}

\section{Dirección Administrativa}
\begin{displayquote}
a) Asistir a la Presidencia en la confección de resoluciones, providencias, ordenanzas, notas, comunicaciones y sus respectivos registros.

b) Administrar los subsidios que se reglamenten y el panteón de la DOSPU.

c) Entender en todo lo relativo a ejecución, registro, identificación y manejo interno de la documentación que ingresa a la DOSPU y DECOM.

d) Organizar y controlar las Actividades/Tareas de vigilancia y limpieza.

f) Desempeñar toda otra actividad que la autoridad le encomiende, en el área de trabajo y de acuerdo con la responsabilidad específica.
\hfill\parencite{CSOrd17}
\end{displayquote}

\subsection{Departamento de Asistencia Técnico-Informática}
\begin{displayquote}
a) Realizar la correcta instalación física de servidores de red y dispositivos de comunicación, el software y la configuración de este, la actualización de los sistemas operativos y dispositivos de conectividad cumpliendo los requerimientos de seguridad informática establecidos para la operación, administración y comunicación de los sistemas y recursos de tecnología de la Universidad.

b) Realizar Actividades/Tareas de mantenimiento preventivo en cableados y configuración de redes, hardware y software de equipos.

c) Confeccionar un registro de todas las fallas críticas producidas en la red de equipamientos.

d) Monitorear la utilización de Internet, correo electrónico y demás tráfico de red.

e) Administrar los usuarios de la red, su seguridad, tanto en los servidores como en los dispositivos de comunicación (routers, switches, y estaciones de trabajo), y del desarrollo de procedimientos de automatización de Actividades/Tareas.

f) Controlar la asignación de privilegios a usuarios.

g) Sugerir medidas a ser implementadas para efectivizar el control de acceso y uso de Internet de los distintos usuarios.

h) Realizar la investigación y puesta en funcionamiento de nuevos clientes de software, y proveer a la capacitación de los agentes para estos fines.

i) Colaborar en la investigación de nuevas tecnologías para redes y comunicaciones, tanto de software como hardware, la confección de los borradores de pliegos de bases y condiciones y la supervisión en las prestaciones de servicios de obras ejecutadas por administración o por terceros.

j) Realizar el soporte a usuarios en el correcto manejo de los dispositivos de hardware y software.

k) Participar en la definición de normas y procedimientos de seguridad a implementar en el ambiente informático.

l) Ejercer el soporte y mantenimiento de la infraestructura de conectividad a la Internet. m) Responder por la instalación y mantenimiento de la estructura de interconexión de las sedes de DOSPU y DECOM.

n) Establecer y fomentar estándares de compras de equipamiento informático.

o) Mantenimiento de la infraestructura eléctrica y de soporte para los servidores

p) Realizar las Actividades/Tareas de diseño, codificación e implementación de los sistemas que se desarrollen en el área.

q) Supervisar la prestación de servicio u obra ejecutada por administración o por terceros en los desarrollos de sistemas.

r) Evaluar la necesidad de nuevos sistemas y sugerir prioridades de desarrollo.

s) Entender en la determinación de los controles y seguridad informática interna.

t) Desempeñar toda otra tarea que la autoridad le encomiende, en el área de trabajo y de acuerdo con la responsabilidad específica.
\hfill\parencite{CSOrd17}
\end{displayquote}

\section{Dirección Médica}
\begin{displayquote}
a) Colaborar con la Presidencia de la DOSPU en la presentación anual del Plan de proyectos a concretar en el área.

b) Efectuar Actividades/Tareas de supervisión médica y odontológica.

c) Supervisar a los Auditores Médicos y Odontológicos.

d) Supervisar a la Farmacia en los aspectos médicos de las compras que se efectúen, formar parte de la Comisión de Vademécum y establecer pautas conjuntas de trabajo.

e) Supervisar a Enfermería.

f) Supervisar la tarea de los profesionales en los Consultorios Internos confeccionando el Organigrama y distribución de los Consultorios e incorporación de profesionales.

g) Intervenir como Asesor en la celebración de convenios con entidades de profesionales de la salud.

h) Participar como Jurado en la selección de profesionales para el servicio interno de la obra social.

i) Ejercer el control y supervisión de las Actividades/Tareas de las Áreas pertenecientes a la Dirección.

j) Organizar la implementación de los planes y programas de prevención.

k) Coordinar la prestación de los servicios en las diferentes sedes.

l) Gestionar el funcionamiento de los servicios propios.

m) Organizar y coordinar el servicio de atención personalizada de los beneficiarios.

n) Recepcionar, tramitar y coordinar la entrega de recetarios de prescripción de medicamentos a los beneficiarios, por el tratamiento de afecciones crónicas y de largo tratamiento.

o) Mantener actualizado el archivo de historias clínicas.

p) Organizar las Actividades/Tareas auxiliares de la medicina y asistencia en los consultorios de prestaciones médicas, odontológicas, quinesiológicas, etc.

q) Desempeñar toda otra actividad que la autoridad le encomiende, en el área de trabajo y de acuerdo con la responsabilidad específica.
\hfill\parencite{CSOrd17}
\end{displayquote}

\subsection{Staff de Auditoría Interna}
\begin{displayquote}
a) Intervenir en la autorización y diligenciamiento de las prestaciones solicitadas por los afiliados.

b) Dar cumplimiento a las normas de calidad de atención médica.

c) Actualización de normas según la evolución de la ciencia médica y leyes nacionales vigentes.
d) Autorizar el Ingreso de Afiliados Adherentes.

e) Comprobar la capacidad estructural de los prestadores, con relación a lo ofrecido.

f) Evaluación de los servicios de internación mediante Auditoria de Terreno. Juntamente de manera inmediatamente elevará informe a Dirección Médica.

g) Auditar, junto a la Dirección de Prestaciones Médicas, la facturación presentada por los prestadores, a los fines de cotejar los valores contractuales, codificación, servicios prestados, calidad de atención médica, etc.

h) Confeccionar indicadores a los fines de detectar desvíos en los servicios prestados.

i) Participar en la confección de convenios con prestadores externos e internos.

j) Análisis de propuestas arancelarias.

k) Control de prestadores asistenciales.

l) Evaluación de autorización de prácticas de Alta Complejidad.

m) Desempeñar toda otra actividad que la autoridad le encomiende, en el área de trabajo y de acuerdo con la responsabilidad específica.
\hfill\parencite{CSOrd17}
\end{displayquote}

\subsection{Departamento de Medicamentos Ambulatorios}
\begin{displayquote}
a) Supervisar el registro de ingreso, archivo y circulación de la documentación que ingrese al área.

b) Elevar informe anual de actividades de la dependencia a su cargo.

c) Mantener registro actualizado de las existencias y necesidades de la renovación del stock de medicamentos ambulatorios.

d) Gestionar las compras de medicamentos ambulatorios, registro y métodos de conservación de todos los medicamentos.

e) Gestionar los registros de medicamentos y métodos de conservación.

f) Realizar controles permanentes del stock y de los registros de adquisición y venta de los medicamentos.

g) Desempeñar toda otra actividad que la autoridad le encomiende, en el área de trabajo y de acuerdo con la responsabilidad específica.
\hfill\parencite{CSOrd17}
\end{displayquote}

\subsection{Departamento Farmacia Medicamentos de Alto Costo}
\begin{displayquote}
a) Supervisar el registro de ingreso, archivo y circulación de la documentación que ingrese al área.

b) Elevar informe anual de actividades de la dependencia a su cargo.

c) Mantener registro actualizado de las existencias y necesidades de la renovación del stock de medicamentos para pacientes crónicos y alto costo.

d) Asistir técnicamente al “Departamento de Servicios Generales, Compras y Contrataciones” en los procesos de compra de los medicamentos de alto costo, diabetes y de los que el Directorio determine por resolución.

e) Gestionar los registros de medicamentos y métodos de conservación.

f) Realizar controles permanentes del stock y de los registros de adquisición y venta de los medicamentos.

g) Desempeñar toda otra actividad que la autoridad le encomiende, en el área de trabajo y de acuerdo con la responsabilidad específica.
\hfill\parencite{CSOrd17}
\end{displayquote}

\section{Dirección Técnico-Contable}
\begin{displayquote}
a) Colaborar en la elaboración de las resoluciones y/o documentos propios de su sector.

b) Entender y supervisar todos los aspectos de financiación, manejo de valores y recursos económicos a cargo de la DOSPU.

c) Responder por el registro y depósito bancario en las cuentas bancarias correspondientes de la totalidad de los ingresos.

d) Supervisar y Refrendar la confección de los balances.

e) Dar cuenta demostrativa de los recursos y gastos y del estado económico-financiero de la Obra Social con referencias estadísticas.

f) Responder por el movimiento patrimonial de la Obra Social.

g) Proponer al Presidente/a de la Obra Social todos los cambios, modificaciones, que considere oportunos para el mejor desenvolvimiento de las Actividades/Tareas y servicios del Departamento.

h) Supervisar el sistema de Afiliaciones y el movimiento de Personal y mesa de entradas.

i) Realizar el registro de las operaciones contables de la obra social.

j) Suscribir a Presidencia el Balance Anual, cuenta de ingreso y egreso anuales.

k) Elevar mensualmente a consideración del Directorio un informe demostrativo de la situación económica-financiera de DOSPU, juntamente con referencias estadísticas que permitan orientar la acción a seguir en base a las posibilidades económicas.

l) Dar cuenta demostrativa de los recursos y gastos y del estado económico-financiero de DOSPU y DECOM con referencias estadísticas.

m) Responder por el registro y depósito bancario en las cuentas bancarias correspondientes de la totalidad de los ingresos y por el movimiento patrimonial de DOSPU y DECOM.

n) Elaborar informes sobre el movimiento patrimonial de la Obra Social. o) Desempeñar toda otra actividad que la autoridad le encomiende, en el área de trabajo y de acuerdo con la responsabilidad específica.
\hfill\parencite{CSOrd17}
\end{displayquote}

\subsection{Departamento Tesorería}
\begin{displayquote}
a) Supervisar el registro de ingreso, archivo y circulación de los expedientes, actuaciones y toda otra documentación que ingrese al área.

b) Colaborar en la elaboración de resoluciones y/o documentos propios de su sector.

c) Recibir recaudaciones diarias, controlarlas y depositarlas.

d) Efectuar los pagos a proveedores, agentes y afiliados.

e) Controlar los ingresos y egresos de la Delegación Villa Mercedes.

f) Realizar el control presupuestario y las rendiciones de cuentas de recursos.

g) Verificar las liquidaciones de aporte a cargo de los afiliados y contribución de todas las dependencias universitarias y controlar sus ingresos.

h) Desempeñar toda otra actividad que la autoridad le encomiende, en el área de trabajo y de acuerdo con la responsabilidad específica.
\hfill\parencite{CSOrd17}
\end{displayquote}

\subsection{Departamento de Compras y Cobranzas}
\begin{displayquote}
a) Entender en las adquisiciones de artículos y materiales para DOSPU y DECOM.

b) Gestionar junto con los Departamentos de Farmacia en la compra de medicamentos de alto costo y diabetes y de todos los medicamentos que el directorio determine por resolución, El Departamento de Farmacia realizará el asesoramiento técnico.

c) Controlar las adquisiciones o ventas que realicen DOSPU y De.COM.

d) Dar curso a los expedientes para la provisión anual de determinados servicios.

e) Controlar la documentación recibida y verificar el cumplimiento de los requisitos exigidos por la normativa vigente.

f) Redactar Contratos, controlar su vigencia, realizar la afectación presupuestaria de los mismos y elevar esta información a las Autoridades.

g) Conservar los contratos y legajos del personal médicos y personal administrativo que haya sido contratado.

h) Colaborar en la presentación de licitaciones, concursos de precios y compras directas.

i) Intervenir en las Comisión Evaluadora y mantener un registro actualizado de compras.

j) Intervenir y coordinar los servicios de seguridad, ambiente, mantenimiento y servicios generales.

k) Supervisar los servicios generales relativos a limpieza y conservación de bienes de DOSPU y DECOM

l) Gestionar las compras de medicamentos de alto costo y diabetes y de todos los medicamentos que el directorio determine por resolución, El Departamento de Farmacia realizara el asesoramiento técnico.

m) Llevar la cartera de cobros y pagos.

n) Desempeñar toda otra actividad que la autoridad le encomiende, en el área de trabajo y de acuerdo con la responsabilidad específica.
\hfill\parencite{CSOrd17}
\end{displayquote}

\subsection{Departamento Administrativo-Contable}
\begin{displayquote}
a) Atención de proveedores, afiliados y entidades financieras.

b) Cargo oportuno de las facturas en el sistema.

c) Control de pagos, tanto a proveedores, como parafiscales e impuestos y tributos. Control de los egresos.

d) Control de facturas y cumplimiento del marco legal vigente, así como también verificar el registro y control de notas de crédito.

e) Realización de notas de débito, conciliaciones bancarias, declaraciones y pagos fiscales.

f) Elaborar libros contables, informes de contabilidad, proyección financiera y flujo de caja, y proyección de los estados financieros.

g) Colaborar en operaciones comerciales (buscar diferentes opciones presupuestarias para cualquier compra o gasto).

h) Colaborar en el manejo del inventario.

i) Desempeñar toda otra actividad que la autoridad le encomiende, en el área de trabajo y de acuerdo con la responsabilidad específica.
\hfill\parencite{CSOrd17}
\end{displayquote}

\section{Dirección de Prestaciones Médicas}
\begin{displayquote}
a) Colaborar en la elaboración de documentos propios del sector.

b) Controlar y constatar las prestaciones y su facturación.

c) Emitir órdenes de pago y registrar los trámites del sistema de reciprocidad.

d) Organizar las Actividades/Tareas auxiliares de los trámites dentro del sistema de reciprocidad.

e) Proponer convenios entre prestadores a los fines de establecer cobertura de las diferentes prácticas.

f) Desempeñar toda otra actividad que la autoridad le encomiende, en el área de trabajo y de acuerdo con la responsabilidad específica.

g) Emitir informes de reciprocidades a pagar a la Dirección Contable y registrar los trámites en el sistema.
\hfill\parencite{CSOrd17}
\end{displayquote}

\subsection{Departamento de Liquidaciones de Prestaciones Médicas}
\begin{displayquote}
a) Facturación de honorarios médicos, anestesistas, cirujanos, bioquímicos, etc.

b) Informe de deuda por parte de los afiliados, prestaciones médicas.

c) Desempeñar toda otra actividad que la autoridad le encomiende, en el área de trabajo y de acuerdo con la responsabilidad específica.
\hfill\parencite{CSOrd17}
\end{displayquote}

\subsection{Departamento de Auditoría Administrativa}
\begin{displayquote}
a) Trabajar de manera conjunta con el Área de Auditoría Médica.

b) Registro de facturación de honorarios y gastos en internación.

c) Control de cruces de facturaciones de internaciones y prácticas ambulatorias.

d) Auditoría de medicación anestésica.

e) Confeccionar expedientes para la solicitud de reintegros frente a otros organismos.

f) Comprobación de valores de prestadores.

g) Comparación de valores entre prestadores.

h) Tramitar ante el Sistema Nacional de Información de Procuración y Trasplante de la República Argentina (SINTRA) toda acción que sea necesaria ante este organismo, o en el futuro el que lo supliere.

i) Desempeñar toda otra actividad que la autoridad le encomiende, en el área de trabajo y de acuerdo con la responsabilidad específica.

j) Confeccionar solicitud y liquidación de reintegros a los afiliados.

k) Realizar las Retenciones Impositivas a cada prestador y proceder a la carga al sistema de AFIP.
\hfill\parencite{CSOrd17}
\end{displayquote}

\subsection{Dirección Administrativa de Villa Mercedes}
\begin{displayquote}
a) Controlar las prestaciones y elevarlas a la Dirección de Prestaciones Médicas de San Luis, para su control y posterior pago.

b) Receptar y dar trámite a las directivas emanadas de la Sede Central.

c) Elaborar periódicamente los informes administrativos-contables y Balances y elevarlos a la Sede Central.

d) Elevar a Presidencia y Directorio el informe mensual de actividades de las dependencias a su cargo.

e) Representar al Presidente en las ocasiones que lo indica la Carta Orgánica.

f) Colaborar con la Dirección General Administrativa en la presentación de informes ante Presidencia y Directorio.

g) Desempeñar toda otra actividad que la autoridad le encomiende, en el área de trabajo y de acuerdo con la responsabilidad específica.
\hfill\parencite{CSOrd17}
\end{displayquote}


\chapter{Categorías y cuotas de afiliación}

\section{Introducción}
Según lo establecido en \cite{CSOrd53}, además del personal docente y no docente de la \acrfull{unsl}, otras personas que cumplan con las condiciones dictadas también pueden afiliarse a \acrshort{dospu}.

A raíz de esto, los afiliados se encuentran dividos en las categorías titular (\cref{sec:titular}), familiar (\cref{sec:familiar}) y voluntario adherente (\cref{sec:adherente}). Asimismo cada categoría está dividida en subcategorías, cada una utilizando distintas fórmulas o coeficientes en las mismas para el cálculo del aporte del afiliado.

\section{Titular} \label{sec:titular}

\subsection{Subcategoria: Obligatorio activo}
\begin{displayquote}
Agentes que se encuentren en actividad.
\hfill\parencite{CSOrd53}, art. 24.1.A.
\end{displayquote}

Actualmente, este descuento no es cálculado por el \acrshort{si}, con lo cual está fuera del alcance de este trabajo.

\subsection{Subcategoria: Voluntario jubilado}
\begin{displayquote}
Agentes que al momento de pasar al régimen pasivo, eran afiliados obligatorios activos y
que opten voluntariamente, mediante formal solicitud por continuar perteneciendo sin
interrupción de aportes a esta Obra Social. La solicitud de afiliación debe ser solicitada en
un término no mayor de sesenta (60) días corridos a partir del cese de sus actividades. \hfill\parencite{CSOrd53}, art. 24.1.B. 
\end{displayquote}

Monto de la cuota (\cite{dospuRes21} art. 2 y Anexo I): $$0.02 * j_m + 0.05 * j_h$$

, donde:
\begin{itemize}
    \item $j_m = \text{jubilación mínima}$
    \item $j_h = \text{haber jubilatorio}$
\end{itemize}

En caso de que dos voluntarios jubilados se tengan vínculo de cónyuge o conviviente, se aplica un descuento del 30\% a la cuota del afiliado (cálculada con la fórmula expuesta para esta subcategoría) con menor haber percibido, quedando como 70\% del monto calculado.

\section{Familiar} \label{sec:familiar}
\begin{displayquote}
\emph{No aportantes o eximidos de aportes}: El cónyuge, los hijos hasta la mayoría de edad establecida por el Código Civil vigente en la República Argentina, o que cursen estudios regulares hasta los 26 años y aquellas personas que pertenezcan al grupo de familiares primarios y están eximidos de aportar por disposición estatutaria o por Resolución de Directorio. No rige límite de edad para aquellos hijos discapacitados mientras dure su incapacidad.
\hfill\parencite{CSOrd53}, art. 23.c.
\end{displayquote}

\begin{displayquote}
Familiares primarios de los titulares.
\hfill\parencite{CSOrd53} art. 24.2.
\end{displayquote}

Distinto es el caso para cónyuges y convivientes de voluntarios jubilados, cuya cuota es un 70 \% de la del afiliado titular (\cite{dospuRes21} art. 2 y Anexo I).

\section{Voluntario adherente} \label{sec:adherente}
\begin{displayquote}
Son aquellos afiliados voluntarios que reúnen los requisitos establecidos en esta Carta Orgánica para ser incorporados a la Obra Social, siendo el Directorio de D.O.S.P.U. quien tiene a su cargo el dictado de las normas de afiliación, pagos de aportes y prestación de servicios correspondiente a cada categoría descrita en esta Carta Orgánica. En todos los casos y para concluir el trámite de afiliación en esta categoría, será requisito la presentación de garantía suficiente en resguardo del pago del aporte correspondiente. Además el Directorio de D.O.S.P.U. reglamentará un cálculo del aporte correspondiente por categoría con valores diferenciados por enfermedades preexistentes o edad avanzada.
\hfill\parencite{CSOrd53} art. 24.3
\end{displayquote}

La \acrfull{cmmu} se define en el 6 \% del sueldo total bruto de un Profesor Universitario Titular Exclusivo con Máxima Antigüedad (\cite{dospuRes21} art. 3).

Según \cite{dospuRes21} art. 3, la cuota de las subcategorías:
\begin{itemize}
    \item becarios y personal ad honorem de la unsl
    \item ascendientes en primer grado del afiliado titular
    \item hijos que hayan dejado de reunir las condiciones de no aportantes
    \item familiares adherentes
    \item universitarios adherentes
    \item ex-afiliados a \acrshort{dospu} 
    \item agentes vinculados a \acrshort{dospu}
\end{itemize}
será un porcentaje sobre el \acrshort{cmmu}. Los porcentajes se encuentran en el Anexo II de la referencia mencionada.

\subsection{Subcategoría: Pensionado}
Monto de la cuota (\cite{dospuRes21} art. 2 y Anexo I): $$0.02 * j_m + 0.05 * p$$

, donde:
\begin{itemize}
    \item $j_m = \text{jubilación mínima}$
    \item $p = \text{pensión}$
\end{itemize}

\subsection{Subcategoría: Agente UNSL con licencia}
El monto de la cuota a abonar por los afiliados de esta categoría es equivalente al monto de los aportes y contribuciones 9\% del sueldo bruto que percibiría como si estuviera en actividad. Asimismo, este monto no puede ser inferior al valor de referencia (CMMU) fijado en \cite{dospuRes21} art. 3 (\cite{dospuRes21} art. 4).

\subsection{Subcategoría: Ascendiente en primer grado}
Para un afiliado de esta categoría con más de diez (10) años de antigüedad en DOSPU, se utiliza como valor de referencia \acrshort{cmmu}20, equivalente a 6\% de un Profesor Universitario Titular Exclusivo con una Antigüedad correspondiente veinte años, en lugar de la \acrshort{cmmu}\cite{dospuRes60}.
Adicionalemente, para afiliados mayores a 66 años de edad, se toma el 150\% en lugar de 200\% del valor de referencia, siendo este último el utilizado para los ascendientes en primer grado sin la antigüedad requerida.

\subsection{Subcategoría: Adherentes de edad avanzada}
El monto a abonar por los afiliados pertenecientes a esta subcategoría depende de si el afiliado tiene 25 o más años de aporte de \acrfull{decom} (\cite{dospuRes7} art. 1.d.):
\begin{itemize}
    \item En caso de tener dicho aporte se toma un 150\% de la \acrshort{cmmu}
    \item En caso contrario se toma un 200\% de la \acrshort{cmmu}.
\end{itemize}

\section{Aportes y seguros adicionales}
Adicionalmente, dependiendo de la categoría y subcategoría del afiliado, el monto de la cuota pued incluir aportes a \acrfull{fesac} y \acrfull{sumas}, así como un seguro en caso de fallecimiento (\cite{dospuRes31}, \cite{dospuRes43} y \cite{dospuRes71}).

\section{Modifcadores de afiliación}
\begin{displayquote}
Establecer distintas categorías de Afiliados Voluntarios Adherentes a los efectos de fijar una cuota mensual diferenciada que deben abonar a la Obra Social \emph{aquellos afiliados que al ingresar}, lo hagan con enfermedades preexistentes o que excedan la edad de 65 años. 
\hfill\parencite{dospuRes7} art. 1.
\end{displayquote}

El modificador se aplica sobre el monto de la cuota calculada, y depende del carácter de la enferdad del afiliado:

\begin{table}[H]
    \centering
\begin{tabular}{|c|c|}
    \hline
    Carácter de la enfermedad & Modificador \\ \hline
    Temporario & 2 \\ \hline
    Crónico & 3 \\ \hline
    De mayor complejidad & 2 \\ \hline
\end{tabular}
\end{table}



\input{arquitectura}

\chapter{Tecnologías empleadas en \SIOSU}

\desarrollar[inline]{Podes completar bastante del trabajo de Vela. Las secciones están propuestas, pero se pueden agregar o modificar.}

\section{Implementación}
\desarrollar[inline]{Tecnologías usadas en la implementación: bd, lógica del negocio, frontend, lenguajes de implementación.}

\section{Desarrollo}

\desarrollar[inline]{Tecnologías usadas durante el desarrollo, por ej. gestión del proyecto, control de versiones, aseguramiento de la calidad, evolución de la base de datos ...}

\section{Despliegue y mantenimiento}

\desarrollar[inline]{Tecnologías usadas para el despliegue y manteniemiento, por ej. para la virtualización y monitoreo (docker y kubernetes).}


\chapter{Motores de reglas}
De forma general, un motor de reglas es una herramienta de software que permite la ejecución de reglas de negocio. Es decir, permite la evaluación de expresiones y la opcional ejecución de acciones y/o retorno de valores, en función de los resultados de la evaluación de expresiones.

Es también de uso común el término Sistema de Gestión de Reglas de Negocio, que a las capacidades mencionadas anteriormente le suma capacidades para la gestión de reglas. En este trabajo, en pos de la brevedad, se utilizarán los términos de forma intercambiable.

\section{Criterio de Evaluación.}
Existen varios proyectos que cuentan con las capacidades descritas en la sección anterior. A continuación, se presentan las características que serán utilizadas para la comparación entre los mismos.

\paragraph{Gestión de las reglas.}
¿Brinda el proyecto herramientas o mecanismos para tareas de la gestión de las reglas, como la creación, modificación, eliminación, evaluación y versionado?

Estas capacidades son necesarias para cumplir con los objetivos de este trabajo, teniendo que ser implementadas en caso de no ser ofrecidas por la herramienta en cuestión.

\paragraph{Integración.}
¿Cómo puede ser el motor integrado con el sistema actual? ¿Cómo se realiza el intercambio de información entre el motor de reglas y el sistema? ¿Cuenta el motor con documentación relevante y actualizada?

Restricciones en los formatos de intercambio de información entre el \acrshort{si} y el motor de reglas imponen también restricciones a la hora de realizar la integración.
Documentación insuficiente o desactualizada resulta en un mayor tiempo requerido y propensidad a errores, siendo lo contrario cierto para documentación completa y actualizada.

\paragraph{Mantenimiento.}
¿Cuenta el proyecto con mantenedores activos? ¿Existen bugs o incompatibilidades que puedan afectar a este trabajo?

\paragraph{Expresividad.}
¿Con qué lenguaje o lenguajes permite el motor la expresión de las reglas? El o los lenguajes deben tener la expresividad suficiente para los cálculos referentes a las cuotas de la obra social.

En concordancia con los objetivos planteados, se prefieren lenguajes que sean inteligibles para personal sin conocimientos específicos de programación.

En pos de facilitar la comparación de los distintos motores, se proveerán ejemplos de reglas que provean funcionalidad correspondiente al código escrito en Java en \cref{lst:calculo_referencia} \cref{lst:interfaz_servicio} y \cref{lst:clase_afiliación}.

\begin{listing}[H]
	\caption{Cálculo de referencia}
	\label{lst:calculo_referencia}
	\inputminted{java}{code/CalculoReferencia.java}
\end{listing}

\begin{listing}[H]
	\caption{Interfaz de servicio}
	\label{lst:interfaz_servicio}
	\inputminted{java}{code/Servicio.java}
\end{listing}

\begin{listing}[H]
	\caption{Clase afiliación}
	\label{lst:clase_afiliación}
	\inputminted{java}{code/Afiliacion.java}
\end{listing}

Dado que el objetivo del ejemplo es ilustrar la expresividad de los lenguajes utilizados por los motores, se obviara código que no sea relevante para la lógica del ejemplo, como getters, setter y código para la compilación de las reglas y su uso desde un programa de Java.
El código completo de los ejemplos funcionales se puede encontrar en \href{https://github.com/IvanB101/ejemplos-motores-reglas}{un respositorio en Github}.

\section{\href{https://github.com/deliveredtechnologies/rulebook}{Rulebook}}
RuleBook es un motor de reglas simple, que está hecho para programadores de Java. Su comportamiento es esperable para los mismos, siendo la definción de reglas en java y su ejecución en orden. Está hecho para simplificar clases inundadas de sentencias \verb|if-else|, permitiendo también el desacoplamiento de las mismas.

\paragraph{Expresividad.}
La expresión de reglas se realiza con el formato Given-When-Then (Dado-Cuando-Entonces), pudiéndose hacer uso de métodos con los nombres correspondientes y funciones lambda o mediante el uso de decoradores en clases y métodos.

\inputminted{java}{code/Rulebook.java}
\captionof{listing}{Ejemplo Rulebook}

\paragraph{Gestión de reglas.}
Dado que las reglas se expresan como código Java, la gestión de las reglas se realiza de igual forma que la gestión del código, con lo cual es independiente del motor.

\paragraph{Integración.}
Dado que las reglas se expresan en código Java, estas pueden hacer uso directo de los objetos de negocio.

\paragraph{Mantenimiento.}
En cuanto al mantenimiento, la última modificación al repositorio correspondiente al motor fue realizada hace más de 4 años.

\section{\href{https://github.com/j-easy/easy-rules}{Easy Rules}}
Easy Rules es un motor de reglas inspirado en un artículo llamado \href{https://martinfowler.com/bliki/RulesEngine.html}{"Should I use a Rules Engine?"} de Martin Fowler. En este el autor dice que para crear un motor de reglas solamente hay que crear objectos con condiciones y acciones, guardarlos en una colleción, y recorrerlos para evaluar las condiciones y ejecutar las acciones.

Easy Rules provee una abstracción para crear reglas con condiciones y acciones y una API para recorrer un conjunto de acciones, evaluando condiciones y ejecutando las acciones que correspondan.

\paragraph{Expresividad.}
Se pueden expresar reglas mediante código Java con anotaciones, o utilizar lenguajes de expresiones, que proveen características de tipado dinámico a aplicaciones o frameworks escritos en java, siendo soportados actualmente MVEL, SpEL y JEXL.

\inputminted{java}{code/EasyRules.java}
\captionof{listing}{Ejemplo Easy Rules}

\paragraph{Gestión de reglas.}
Al igual que Rulebook, la gestión de reglas se realiza gestionando el código correspondiente a las mismas, sea Java o alguno de los lenguajes de expresión. De nuevo, el motor no brinda herramientas para este proposito.

\paragraph{Integración.}
Independientemente del lenguaje que se utilice para la expresión de las reglas, las mismas pueden hacer uso directo de los objectos de negocio.

\paragraph{Mantenimiento.}
Desde diciembre de 2020, este proyecto se encuentra en modo mantenimiento, soportando únicamente la versión 4.1.x y solamente realizando cambios para la corrección de errores.

\section{\href{http://alvarestech.com/temp/fuzzyjess/Jess60/Jess70b7/docs/index.html}{jess}}
Jess es un motor de reglas para la plataforma de Java, en dicho lenguaje de programación. Fue desarrollado por Ernest Friedman-Hill, publicado en 1995, diseñado para la automatización de sistemas expertos.

Jess utiliza un paradigma declarativo, donde las reglas son continuamente aplicadas a un conjunto de hechos utilizando búsqueda de patrones con el algoritmo Rete. Las reglas puden alterar el conjunto de hechos o ejecutar código Java arbitrario.

\paragraph{Expresividad.}
Para definir las reglas se puede hacer uso del lenguaje de reglas Jess o XML, pudiéndose además proveer datos a las reglas para la manipulación de las mismas

\paragraph{Gestión de reglas.}
La gestión de las reglas se consigue mediante la gestión de los archivos con XML o el lenguaje de reglas de Jess. Existen varios plugins para la IDE Eclipse que lo dotan de todas las capacidades de un editor de código para el lenguaje de reglas de Jess.

\paragraph{Integración.}
Las reglas pueden hacer uso directo de los objetos de negocio. Adicionalmente, el motor puede ser ejecutado como un programa independiente o ser integrado dentro de otro.

\paragraph{Mantenimiento.}
La última versión de Jess fue publicada en marzo de 2006.

\section{\href{https://www.drools.org/}{Drools}}
Drools es un sistema de gestión de reglas de negocio, con un motor de reglas basado en inferencia de encandenamiento progresivo y regresivo. Drools es parte de Apache KIE y soporta la API de reglas de Java, specificada en \href{https://www.oracle.com/technical-resources/articles/javase/javarule.html}{JSR 94} (siglas de Java Specification Request, solicitud de especificación de Java).

\paragraph{Expresividad.}
Para escribir reglas se puede utilizar \acrcomplete{drl}, o utilizando el \acrcomplete{dmn}.

En caso de utilizar el segundo, para la representación visual de las reglas se utiliza un \acrcomplete{drd}.

Las decisiones, componentes de las reglas donde se encuentra la lógica, están conformadas por tablas, cuyas celdas contienen lógica expresada en \acrcomplete{feel}.

\begin{center}
	\begin{figure}
		\includegraphics[width=3cm]{drools_diagram.png}
	\end{figure}
\end{center}

\begin{center}
	\begin{figure}
		\includegraphics*[viewport=0 550 789 1175, width=1.3\textwidth]{drools_table.png}
		\includegraphics*[viewport=0 0 789 550, width=1.3\textwidth]{drools_table.png}
	\end{figure}
\end{center}

\paragraph{Gestión de reglas.}
Drools provee un editor visual para la edición de reglas que utilizan \acrshort{dmn}. El editor es una librería web autocontenida, también existe una extensión para Visual Studio Code, que expone la funcionalidad de la misma.

\paragraph{Integración.}
Las reglas de Drools pueden hacer uso de \acrcomplete{pojo}, es decir objetos con campos, getters, setters y constructores, funcionalidad que no pudo ser utilizada en el ejemplo por los problemas mencionados en \cref{para:drools_mantenimiento}.

Alternativamente, los tipos pueden ser definidos utilizando en el editor visual, pasando luego mapas con los valores correspondientes, siendo las claves los nombres de los campos.

\paragraph{Mantenimiento.}\label{para:drools_mantenimiento}
Este motor de reglas es open source y de uso libre, y su desarrollo continúa actualmente activo (hasta la última fecha de edición de esta sección, 20/03/25, la última versión fue publicada el 28/04/24).

Utilizando Ubuntu 22.04.5 LTS, tanto en la extensión como la librería web, se notaron problemas a la hora de importar de otros modelos o de clases Java. Los archivos \verb|.dmn| no eran encontrados por la extensión, mientras que la opción para importar clases Java no estaba disponible. Por otro lado, al cambiar los tipos de las expresiones (Feel Expresion en el ejemplo arriba) al finalizar la edición de uno, los otros volvía a tener el valor inicial de <Undefined>, que es la razón de que tengan ese valor en el ejemplo anterior.

Adicionalmente, incluyendo la librería en un sitio web, únicamente instanciando el componente editor de la librería resulta en numerosos mensajes de advertencia en la consola.

\section{\href{https://openl-tablets.org/}{OpenL Tablets}}
OpenL Tablets es un sistema de gestión de reglas y un motor de reglas basado en la representación de reglas como tablas, desarrollado por el \href{https://openl-tablets.org/community/team}{equipo de OpenL}. Su desarrollo comenzó en 2003 como un projecto interno y fue luego publicado en 2006 en SourceForge. En un principio era únicamente un motor de reglas, convirtiendose en un sistema de gestión de reglas a partir de la versión 5.

\paragraph{Expresividad.}
Para la definición de las reglas se utilizan tablas basadas en Microsoft Excel. Dentro de estas celdas pueden haver valores, acciones o expressiones. Las últimas dos hacen uso de BEX, que es una extensión de la gramática de Java.

\begin{table}[!ht]
	\makeatletter\@ifundefined{tablewidth}{\newlength{\tablewidth}}{}
\makeatother\setlength{\tablewidth}{\dimexpr \textwidth - 3\arrayrulewidth - 6\tabcolsep \relax}
\setlength{\extrarowheight}{-5pt}

\begin{tabular}{|p{0.21\tablewidth}|p{0.27\tablewidth}|p{0.51\tablewidth}|}
		\hline
		\multicolumn{3}{|C{{\dimexpr 1.0\tablewidth + 2\arrayrulewidth + 4\tabcolsep \relax}}|}{\color[HTML]{FFFFFF}\cellcolor[HTML]{000000}Rules Double CalcularCuota(AfiliacionWrapper afiliado)} \\ \hline
		\color[HTML]{000000}\cellcolor[HTML]{CCFFFF}C1
		 & \color[HTML]{000000}\cellcolor[HTML]{CCFFFF}C2
		 & \color[HTML]{000000}\cellcolor[HTML]{CCFFFF}RET1                                                                                                                                         \\ \hline
		\color[HTML]{000000}\cellcolor[HTML]{CCFFFF}categoria
		 & \color[HTML]{000000}\cellcolor[HTML]{CCFFFF}subcategoria
		 & \color[HTML]{000000}\cellcolor[HTML]{CCFFFF}value                                                                                                                                        \\ \hline
		\cellcolor[HTML]{CCFFFF}
		 & \cellcolor[HTML]{CCFFFF}
		 & \color[HTML]{000000}\cellcolor[HTML]{CCFFFF}Double value                                                                                                                                 \\ \hline
		\color[HTML]{000000}\cellcolor[HTML]{FFFF99}Categoria
		 & \color[HTML]{000000}\cellcolor[HTML]{FFFF99}Sub Categoria
		 & \color[HTML]{000000}\cellcolor[HTML]{FFCC99}Fee Value                                                                                                                                    \\ \hline
		\cellcolor[HTML]{FFFF99}
		 & \color[HTML]{000000}\cellcolor[HTML]{FFFF99}CONYUGE
		 & \color[HTML]{000000}\cellcolor[HTML]{FFCC99}=CalcularCuota(conyuge) * 0.7\strut                                                                                                          \\\cline{2-3} \noalign{\vskip 0.5pt}
		\multirow{-2}{0.21\tablewidth}{\color[HTML]{000000}\cellcolor[HTML]{FFFF99}FAMILIAR}
		 & \color[HTML]{000000}\cellcolor[HTML]{FFFF99}DESCENDIENTE\allowbreak\_PRIMER\allowbreak\_GRADO
		 & \color[HTML]{000000}\cellcolor[HTML]{FFCC99}0                                                                                                                                            \\ \hline
		\cellcolor[HTML]{FFFF99}
		 & \color[HTML]{000000}\cellcolor[HTML]{FFFF99}BECARIO
		 & \color[HTML]{000000}\cellcolor[HTML]{FFCC99}=CMMU * ModificadorBecario(afiliado)\strut                                                                                                   \\\cline{2-3} \noalign{\vskip 0.5pt}
		\multirow{-2}{0.21\tablewidth}{\color[HTML]{000000}\cellcolor[HTML]{FFFF99}VOLUNTARIO\allowbreak\_ADHERENTE}
		 & \color[HTML]{000000}\cellcolor[HTML]{FFFF99}AGENTE\allowbreak\_UNSL\allowbreak\_CON\allowbreak\_LICENCIA
		 & \color[HTML]{000000}\cellcolor[HTML]{FFCC99}=afiliado.getHaberPercibido() * ModificadorAgente(afiliado)                                                                                  \\ \hline
		\cellcolor[HTML]{FFFF99}
		 & \cellcolor[HTML]{FFFF99}
		 & \color[HTML]{000000}\cellcolor[HTML]{FFCC99}=error(``categoria desconocida'')                                                                                                            \\ \hline
	\end{tabular}

	\caption{Tabla principal del ejemplo}
\end{table}

\begin{table}[!ht]
	\makeatletter\@ifundefined{tablewidth}{\newlength{\tablewidth}}{}
\makeatother\setlength{\tablewidth}{\dimexpr \textwidth - 2\arrayrulewidth - 4\tabcolsep \relax}
\setlength{\extrarowheight}{-5pt}

\begin{tabular}{|p{0.33\tablewidth}|p{0.67\tablewidth}|}
	\hline
	\multicolumn{2}{|C{{\dimexpr 1.0\tablewidth + 1\arrayrulewidth + 2\tabcolsep \relax}}|}{\color[HTML]{FFFFFF}\cellcolor[HTML]{000000}Environment} \\ \hline
	\color[HTML]{000000}\cellcolor[HTML]{FFFF99}import
	 & \color[HTML]{000000}\cellcolor[HTML]{FFFF99}com.example.afiliacion\allowbreak\_wrapper                                                        \\ \hline
\end{tabular}

	\caption{Importación de las clases de Java}
	\label{tab:openl_java_import}
\end{table}

\begin{table}[!ht]
	\makeatletter\@ifundefined{tablewidth}{\newlength{\tablewidth}}{}
\makeatother\setlength{\tablewidth}{\dimexpr \textwidth - 3\arrayrulewidth - 6\tabcolsep \relax}
\setlength{\extrarowheight}{-5pt}

\begin{tabular}{|p{0.28\tablewidth}|p{0.25\tablewidth}|p{0.47\tablewidth}|}
	\hline
	\multicolumn{3}{|C{{\dimexpr 1.0\tablewidth + 2\arrayrulewidth + 4\tabcolsep \relax}}|}{\color[HTML]{FFFFFF}\cellcolor[HTML]{000000}Rules Double ModificadorBecario(AfiliacionWrapper afiliado)} \\ \hline
	\multicolumn{2}{|C{{\dimexpr 0.5330525030525031\tablewidth + 1\arrayrulewidth + 2\tabcolsep \relax}}|}{\color[HTML]{000000}\cellcolor[HTML]{CCFFFF}C3}
	 & \color[HTML]{000000}\cellcolor[HTML]{CCFFFF}RET1                                                                                                                                              \\ \hline
	\multicolumn{2}{|C{{\dimexpr 0.5330525030525031\tablewidth + 1\arrayrulewidth + 2\tabcolsep \relax}}|}{\color[HTML]{000000}\cellcolor[HTML]{CCFFFF}edad}
	 & \color[HTML]{000000}\cellcolor[HTML]{CCFFFF}value                                                                                                                                             \\ \hline
	\color[HTML]{000000}\cellcolor[HTML]{CCFFFF}int min
	 & \color[HTML]{000000}\cellcolor[HTML]{CCFFFF}int max
	 & \color[HTML]{000000}\cellcolor[HTML]{CCFFFF}Double value                                                                                                                                      \\ \hline
	\color[HTML]{000000}\cellcolor[HTML]{FFFF99}Desde
	 & \color[HTML]{000000}\cellcolor[HTML]{FFFF99}Hasta
	 & \color[HTML]{000000}\cellcolor[HTML]{FFCC99}Modificador                                                                                                                                       \\ \hline
	\cellcolor[HTML]{FFFF99}
	 & \color[HTML]{000000}\cellcolor[HTML]{FFFF99}30
	 & \color[HTML]{000000}\cellcolor[HTML]{FFCC99}0.45                                                                                                                                              \\ \hline
	\color[HTML]{000000}\cellcolor[HTML]{FFFF99}31
	 & \color[HTML]{000000}\cellcolor[HTML]{FFFF99}45
	 & \color[HTML]{000000}\cellcolor[HTML]{FFCC99}0.82                                                                                                                                              \\ \hline
	\color[HTML]{000000}\cellcolor[HTML]{FFFF99}46
	 & \color[HTML]{000000}\cellcolor[HTML]{FFFF99}55
	 & \color[HTML]{000000}\cellcolor[HTML]{FFCC99}1                                                                                                                                                 \\ \hline
	\color[HTML]{000000}\cellcolor[HTML]{FFFF99}56
	 & \color[HTML]{000000}\cellcolor[HTML]{FFFF99}65
	 & \color[HTML]{000000}\cellcolor[HTML]{FFCC99}1.2                                                                                                                                               \\ \hline
	\color[HTML]{000000}\cellcolor[HTML]{FFFF99}66
	 & \cellcolor[HTML]{FFFF99}
	 & \color[HTML]{000000}\cellcolor[HTML]{FFCC99}2                                                                                                                                                 \\ \hline
\end{tabular}

	\caption{Tabla modificador becario}
\end{table}

\begin{table}[!ht]
	\makeatletter\@ifundefined{tablewidth}{\newlength{\tablewidth}}{}
\makeatother\setlength{\tablewidth}{\dimexpr \textwidth - 3\arrayrulewidth - 6\tabcolsep \relax}
\setlength{\extrarowheight}{-5pt}

\begin{tabular}{|p{0.23\tablewidth}|p{0.24\tablewidth}|p{0.53\tablewidth}|}
	\hline
	\multicolumn{3}{|C{{\dimexpr 0.9999999999999999\tablewidth + 2\arrayrulewidth + 4\tabcolsep \relax}}|}{\color[HTML]{FFFFFF}\cellcolor[HTML]{000000}Rules Double ModificadorAgente(AfiliacionWrapper afiliado)} \\ \hline
	\multicolumn{2}{|C{{\dimexpr 0.4692429792429792\tablewidth + 1\arrayrulewidth + 2\tabcolsep \relax}}|}{\color[HTML]{000000}\cellcolor[HTML]{CCFFFF}C3}
	 & \color[HTML]{000000}\cellcolor[HTML]{CCFFFF}RET1                                                                                                                                                            \\ \hline
	\multicolumn{2}{|C{{\dimexpr 0.4692429792429792\tablewidth + 1\arrayrulewidth + 2\tabcolsep \relax}}|}{\color[HTML]{000000}\cellcolor[HTML]{CCFFFF}edad}
	 & \color[HTML]{000000}\cellcolor[HTML]{CCFFFF}value                                                                                                                                                           \\ \hline
	\color[HTML]{000000}\cellcolor[HTML]{CCFFFF}int min
	 & \color[HTML]{000000}\cellcolor[HTML]{CCFFFF}int max
	 & \color[HTML]{000000}\cellcolor[HTML]{CCFFFF}Double value                                                                                                                                                    \\ \hline
	\color[HTML]{000000}\cellcolor[HTML]{FFFF99}Desde
	 & \color[HTML]{000000}\cellcolor[HTML]{FFFF99}Hasta
	 & \color[HTML]{000000}\cellcolor[HTML]{FFCC99}Modificador                                                                                                                                                     \\ \hline
	\cellcolor[HTML]{FFFF99}
	 & \color[HTML]{000000}\cellcolor[HTML]{FFFF99}30
	 & \color[HTML]{000000}\cellcolor[HTML]{FFCC99}0.4                                                                                                                                                             \\ \hline
	\color[HTML]{000000}\cellcolor[HTML]{FFFF99}31
	 & \color[HTML]{000000}\cellcolor[HTML]{FFFF99}45
	 & \color[HTML]{000000}\cellcolor[HTML]{FFCC99}0.78                                                                                                                                                            \\ \hline
	\color[HTML]{000000}\cellcolor[HTML]{FFFF99}46
	 & \color[HTML]{000000}\cellcolor[HTML]{FFFF99}55
	 & \color[HTML]{000000}\cellcolor[HTML]{FFCC99}0.95                                                                                                                                                            \\ \hline
	\color[HTML]{000000}\cellcolor[HTML]{FFFF99}56
	 & \color[HTML]{000000}\cellcolor[HTML]{FFFF99}65
	 & \color[HTML]{000000}\cellcolor[HTML]{FFCC99}1.13                                                                                                                                                            \\ \hline
	\color[HTML]{000000}\cellcolor[HTML]{FFFF99}66
	 & \cellcolor[HTML]{FFFF99}
	 & \color[HTML]{000000}\cellcolor[HTML]{FFCC99}1.86                                                                                                                                                            \\ \hline
\end{tabular}

	\caption{Tabla secundaria modificador agente UNSL}
\end{table}

\paragraph{Gestión de reglas.}
El motor posee una aplicación web para la creación, modificación, eliminación, prueba y versionado. Además, se separan los repositorios de diseño y producción, permitiendo la colaboración en edición de reglas y el versionado de las mismas de forma independiente del \acrshort{si}. Alternativamente, pueden utilizarse Microsoft Excel, of software similar como Libre Office, para la creación, modificación y eliminación de las reglas.

\paragraph{Integración.}
Por otra parte, OpenL Tablets posee otra aplicación para exponer las reglas almacenadas en un repositorio como servicios web. Alternativamente, brinda clases de utilidad para la compilación de las reglas. Estas utilizan las tablas para generar clases wrapper, las cuales también pueden ser específicadas manualmente. En caso de utilzar las clases generadas, las reglas pueden accederse por medio de la reflección, siendo métodos de las clases con los mismos nombres de las reglas.

Las reglas de OpenL pueden hacer uso directo de las instancias de las clases Java. Para esto, se utiliza una tabla para importar las mismas, como la presente en \cref{tab:openl_java_import}.

\paragraph{Mantenimiento.}
Este motor de reglas es open source y de uso libre, y su desarrollo continúa actualmente activo (hasta la última fecha de edición de esta sección, 20/03/25, la última versión fue publicada el 28/02/2025).



\chapter{Relevamiento código fuente}


\chapter{Traducción al paradigma funcional}

\section{Caso de estudio 1}

\desarrollar[inline]{Toma el ejemplo más simple desarrollado y explicas como lo pasaste a funcional}

\section{Caso de estudio 2}

\section{Caso de estudio 3}

\chapter{Parametrización en reglas}
\label{chap:solucion}

\section{Elección del Motor a Utilizar}
Como se mencionó anteriormente, el objectivo de este trabajo es facilitar la gestión de las reglas de un \SIOSU, particularmente por personas sin o con reducidos conocimientos relacionados con la programación. Teniendo esto en cuenta, se considera que, irrespectivamente de los demás aspectos considerados en el criterio de decisión, Drools y OpenL Tablets resultan las únicas opciones de interés para este trabajo. La siguiente es una comparación más detallada de estas opciones:

\paragraph{Expresividad.}
La expresividad que ambos proveen para las reglas resulta comparable. Dicho esto, las tablas de Excel probablemente resulten más familiares para la mayoría de personas que los diagramas DMN.

\paragraph{Gestión de reglas.}
En este aspecto, OpenL Tablets resulta superior, ya que cuenta con herramientas para el versionado de las reglas, además del resto de capacidades que comparte con Drools, como la creación, modificación y eliminación de reglas.

\paragraph{Integración.}
En este campo los motores son similares, por lo menos en su especificación. Dados los problemas mencionados en la subsección de Drools del cápitulo anterior, estás capacidades no pudieron ser probadas para este.

\paragraph{Mantenimiento.}
Ambos proyectos son actualemente (hasta la última fecha de edición de esta sección, 08/04/2025) mantenidos.

Por otra parte, mientras que en el caso de OpenL Tablets no se encontró ningún problema durante la elaboración del ejemplo, lo mismo no se puede decir de Drools. Con este último se encontraron varios problemas, mencionados en la sección correspondiente del cápitulo anterior, relacionados con las herramientas para la creación y edición de las reglas.

\paragraph{Elección final.}
Teniendo en cuenta este análisis, se puede ver que, en los aspectos que resultan de interés para este trabajo, OpenL Tablets resulta igual o superior en cada uno. Consecuentemente, se hará uso del mismo de aquí en adelante.

\section{Caso de estudio 1}

\desarrollar[inline]{Toma el ejemplo más simple desarrollado y explicas como lo parametrizaste con el motor de reglas.}


\section{Caso de estudio 2}

\section{Caso de estudio 3}

\input{conclusiones}

\backmatter

\printbibliography

\end{document}
